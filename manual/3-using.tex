\chapter{Using Nomad}

In this chapter we review the basics of using \nomad to in practice, 
from installation to computing differential operations to implementing
new functions into the language.

\section{Installation}

\nomad is a header-only library, meaning there are no shared objects
to compile and build before compiling an executable itself.  Note that
this also means that any attempt to link shared objects that have each
included \nomad headers will fail.

When incorporating \nomad into an executable there are two
dependencies.  Firstly, \nomad requires a compiler supporting C++11 
\textbf{reference}, including but not limited to \verb|clang++ 3.4|,
\verb|g++ 4.8|, and above.  The second is Eigen, the header-only 
library implementing a wide array of efficient linear algebra
routines \textbf{reference}.

For example, if you wish to incorporate \nomad into an executable
that compiles as
%
\begin{verbatim}
./compiler -03 -o main main.cpp
\end{verbatim}
%
then including \nomad nominally requires only the addition of the 
appropriate paths,
%
\begin{verbatim}
./compiler -03 -I/path/to/Nomad -I/path/to/Eigen -o main main.cpp
\end{verbatim}
%
Note that some compilers do not implement C++11 by default.  In
this case you may have to explicit enable C++11 and the updated
Standard Library with
%
\begin{verbatim}
./compiler -03 -std=c++11 -stdlib=libc++ -I/path/to/Nomad -I/path/to/Eigen -o main main.cpp
\end{verbatim}

\section{Computing Differential Operations}

As noted in Section \ref{sec:user_interface}, linear differential operations 
are in \nomad are implemented using a functor-functional pattern, with 
functions wrapped in functors and differential operations implemented 
as functionals.

Here a functor is a class deriving from \textbf{base functor} and implementing
the public member function
%
\begin{verbatim}
T operator()(const Eigen::VectorXd& x) const,
\end{verbatim}
%
where \verb|T| is any automatic differentiation variable.  For example,
if we wanted to compute the gradient of the function
%
\begin{verbatim}
double example_function(const Eigen::VectorXd& x) {
  double x1 = x[0];
  double x2 = x[1];
  double x3 = x[2];
  return exp( cos(x1) + sin(x2) ) / x3;
}
\end{verbatim}
%
we would need to define the class \textbf{update to base functor derivation}
%
\begin{verbatim}
class example_functor {
  var<1U> operator()(const Eigen::VectorXd& x) const {
    var<1U> x1 = x[0];
    var<1U> x2 = x[1];
    var<1U> x3 = x[2];
    return exp( cos(x1) + sin(x2) ) / x3;
  }
};
\end{verbatim}
%
The full list of functions that have been implemented for automatic
differentiation variables can be found in Section \ref{sec:operators_and_functions}.

Specifying a functor for each automatic differentiation variable can be
burdensome and it is usually easier to define a template class.  For
example, the above would become
%
\begin{verbatim}
template <typename T>
class example_functor {
  T operator()(const Eigen::VectorXd& x) const {
    T x1 = x[0];
    T x2 = x[1];
    T x3 = x[2];
    return exp( cos(x1) + sin(x2) ) / x3;
  }
};
\end{verbatim}
%
with
%
\begin{verbatim}
example_functor<var<1U>>
example_functor<var<2U>>
example_functor<var<3U>>
\end{verbatim}
%
defining classes for first, second, and third-order automatic differentiation
variables, respectively.  \textbf{Use typdefs and refer to typedef table 
in previous chapter}.

Differential operators are implemented as functionals which take
the functors as arguments.  A gradient, for example, can be computed
as
%
\begin{verbatim}
Eigen::VectorXd& x = ...;
Eigen::VectorXd& g = Eigen::VectorXd::Zero(x.size());
gradient(example_functor<var<1U>>, x, g);
\end{verbatim}
%
and the Hessian as
%
\begin{verbatim}
Eigen::VectorXd& x = ...;
Eigen::VectorXd& H = Eigen::VectorXd::Zero(x.size(), x.size());
hessian(example_functor<var<2U>>, x, H);
\end{verbatim}
%
All linear differential operations implanted in \nomad are detailed in \ref{sec:differential_operations}.

\section{Implementing Functions}

Although \nomad implements a diverse array of functions, difficult applications
may require the addition of new functions specific to the problem at hand.  The
implementation of new functions in \nomad is straightforward.

The biggest complication in implementing a new function is in properly
managing the global stack that represents the expression graph.  In particular,
a function has to define the number of inputs and partials it will be using
so that the stack can expand if necessary, create a node on the expression
graph for the function, and then push the necessary inputs, dual numbers,
and partials onto the stack.  For details on how the stacks on the implemention
see Chapter \ref{chp:implementation}.

\subsection{Generic Unary Function}

A unary function, for example, might be implemented in \verb|new_function.hpp|
as \textbf{Need all the template configuration!}
%
\begin{verbatim}
#ifndef new_funcion_hpp
#define new_function_hpp

#include <var/var.hpp>

namespace nomad {

  inline double new_function(double x) { return ...; }
  
  template <short autodiff_order>
  inline var<autodiff_order> new_function(const var<autodiff_order>& input) {
    
    const short partials_order = 3;
    const unsigned int n_inputs = 1;
    
    next_inputs_delta = n_inputs;
    next_partials_delta =
      var_body<autodiff_order, partials_order>::n_partials(n_inputs);
    
    new var_body<autodiff_order, partials_order>(n_inputs);

    const double x = input.first_val();
    push_dual_numbers<autodiff_order>(new_function(x));
    
    push_inputs(input.dual_numbers());
    
    if (autodiff_order >= 1) push_partials(df_dx);
    if (autodiff_order >= 2) push_partials(d2f_dx2);
    if (autodiff_order >= 3) push_partials(d3f_dx3);

    return var<autodiff_order>(next_body_idx_ - 1);
  }
}
#endif
\end{verbatim}

Firstly we define the double-valued function,
%
\begin{verbatim}
 double new_function(double x).
\end{verbatim}
%
although this isn't strictly necessary if the double-valued function is defined
in an external namespace, it can be helpful to ensure consistency when
calling the function from within the \verb|nomad| namespace.

Next we implement the automatic differentiation viable-valued function,
%
\begin{verbatim}
  template <short autodiff_order>
  inline var<autodiff_order> new_function(const var<autodiff_order>& input)
\end{verbatim}
%
Note the use of templates to avoid defining the function for each different
automatic differentiation variable.  The first critical step in defining this
function is informing the stack about how much space the function will need,
%
\begin{verbatim}
    const short partials_order = 3;
    const unsigned int n_inputs = 1;
    
    next_inputs_delta = n_inputs;
    next_partials_delta =
      var_body<autodiff_order, partials_order>::n_partials(n_inputs);
\end{verbatim}
%
With the stack ready we can create a new node for the function on the
expression graph,
%
\begin{verbatim}
    new var_body<autodiff_order, partials_order>(n_inputs);
\end{verbatim}
%
Here \verb|operator new| has been overloaded, so that this call creates
a new node on the top of the node stack.

With the node pushed onto the stack we just have to set the dual numbers,
inputs, and partials,
%
\begin{verbatim}
    const double x = input.first_val();
    push_dual_numbers<autodiff_order>(new_function(x));
    
    push_inputs(input.dual_numbers());
    
    if (autodiff_order >= 1) push_partials(df_dx);
    if (autodiff_order >= 2) push_partials(d2f_dx2);
    if (autodiff_order >= 3) push_partials(d3f_dx3);
\end{verbatim}
%
before returning a new automatic differentiation variable pointing to the new
node,
%
\begin{verbatim}
    return var<autodiff_order>(next_body_idx_ - 1);
\end{verbatim}

If not all partial derivatives can be implemented for the new function then the
higher-order implementations should throw an exception.  For example, we
specify that we have only first-order partials and then throw exceptions for
if second or third-order partials are requested,
%
\begin{verbatim}
#ifndef new_funcion_hpp
#define new_function_hpp

#include <var/var.hpp>
#include <autodiff/exceptions.hpp>

namespace nomad {

  inline double new_function(double x) { return ...; }
  
  template <short autodiff_order>
  inline var<autodiff_order> new_function(const var<autodiff_order>& input) {
    
    const short partials_order = 1;
    const unsigned int n_inputs = 1;
    
    next_inputs_delta = n_inputs;
    next_partials_delta =
      var_body<autodiff_order, partials_order>::n_partials(n_inputs);
    
    new var_body<autodiff_order, partials_order>(n_inputs);

    const double x = input.first_val();
    push_dual_numbers<autodiff_order>(new_function(x));
    
    push_inputs(input.dual_numbers());
    
    if (autodiff_order >= 1) push_partials(df_dx);
    if (autodiff_order >= 2) {
      throw partial_fail_ex("Only first-order partials have "
                            "been implemented for new_function");
    }
    if (autodiff_order >= 3) {
      throw partial_fail_ex("Only first-order partials have "
                            "been implemented for new_function");
    }

    return var<autodiff_order>(next_body_idx_ - 1);
  }
}
#endif
\end{verbatim}

\subsection{Generic Binary Function}

The same pattern holds if we include a binary function, only now we have
to consider all three-possible combinations of double and automatic
differentiation variable arguments,
%
\begin{verbatim}
#ifndef new_binary_function_hpp
#define new_binary_function_hpp

#include <math.h>
#include <var/var.hpp>

namespace nomad {
  
  inline double new_binary_function(double x, double y) {
    return ...;
  }
  
  template <short autodiff_order>
  inline var<autodiff_order> new_binary_function(const var<autodiff_order>& v1,
                                                 const var<autodiff_order>& v2) {
    
    const short partials_order = 3;
    const unsigned int n_inputs = 2;
    
    next_inputs_delta = n_inputs;
    next_partials_delta =
      var_body<autodiff_order, partials_order>::n_partials(n_inputs);
    
    new var_body<autodiff_order, partials_order>(n_inputs);

    double x = v1.first_val();
    double y = v2.first_val();
    
    push_dual_numbers<autodiff_order>(new_binary_function(x, y));
    
    push_inputs(v1.dual_numbers());
    push_inputs(v2.dual_numbers());

    if (autodiff_order >= 1) {
      push_partials(df_dx);
      push_partials(df_dy);
    }
    if (autodiff_order >= 2) {
      push_partials(d2f_dx2);
      push_partials(d2f_dxdy);
      push_partials(d2f_dy2);
    }
    if (autodiff_order >= 3) {
      push_partials(df3_dx3);
      push_partials(df3_dx2dy);
      push_partials(df3_dxdy2);
      push_partials(df3_dy3);
    }

    return var<autodiff_order>(next_body_idx_ - 1);
    
  }
  
  template <short autodiff_order>
  inline var<autodiff_order> new_binary_function(double x,
                                                 const var<autodiff_order>& v2) {
    
    const short partials_order = 3;
    const unsigned int n_inputs = 1;
    
    next_inputs_delta = n_inputs;
    next_partials_delta =
      var_body<autodiff_order, partials_order>::n_partials(n_inputs);
    
    new var_body<autodiff_order, partials_order>(n_inputs);
    
    double y = v2.first_val();
    
    push_dual_numbers<autodiff_order>(new_binary_function(x, y));
    
    push_inputs(v2.dual_numbers());
    
    if (autodiff_order >= 1) push_partials(df_dy);
    if (autodiff_order >= 2) push_partials(df2_dy2);
    if (autodiff_order >= 3) push_partials(df3_dy3);
    
    return var<autodiff_order>(next_body_idx_ - 1);
    
  }
  
  template <short autodiff_order>
  inline var<autodiff_order> new_binary_function(const var<autodiff_order>& v1,
                                                 double y) {
    
    const short partials_order = 3;
    const unsigned int n_inputs = 1;
    
    next_inputs_delta = n_inputs;
    next_partials_delta =
      var_body<autodiff_order, partials_order>::n_partials(n_inputs);
    
    new var_body<autodiff_order, partials_order>(n_inputs);
    
    double x = v1.first_val();
    
    push_dual_numbers<autodiff_order>(new_binary_function(x, y));
    
    push_inputs(v1.dual_numbers());
    
    if (autodiff_order >= 1) push_partials(df_dx);
    if (autodiff_order >= 2) push_partials(d2f_dx2);
    if (autodiff_order >= 3) push_partials(d3f_dx3);
    
    return var<autodiff_order>(next_body_idx_ - 1);
    
  }

}
#endif
\end{verbatim}

\subsection{Custom Node Types}

As noted in Section \ref{sec:exp_graph_rep}, some functions may benefit
from optimized propagation routines defined in custom \verb|var_body|
implementations.  For example, \nomad defines special unary and binary
function types with optimized propagation functions.  Using these the binary
example above would become

Defining new \verb|var_body| specializations is a substantial task and anyone
interesting in introducing a new specialization should contact the Stan Development
Team directly.