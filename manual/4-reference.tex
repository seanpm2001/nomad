\chapter{Nomad Reference Guide} \label{chap:reference_guide}

\section{Linear Differential Operations} \label{sec:differential_operations}

\subsection{Zeroth-Order Methods}

This methods requires a functor implementing
%
\begin{verbatim}
double operator()(const Eigen::VectorXd& x) const
\end{verbatim}

\begin{tcolorbox}[colback=white,colframe=gray90, coltitle=black,boxrule=3pt,
fonttitle=\bfseries,title=Finite Difference Gradient]

\begin{tabular}{llll}
\textit{I/O} & \textit{Type} & \textit{Parameter} & \textit{Description} \\
Input & \texttt{const F\&} & \texttt{f} & Functor Implementation \\
Input & \texttt{const Eigen::VectorXd\&} & \texttt{x} & Input Values \\
Output & \texttt{Eigen::VectorXd\&} & \texttt{g} & Gradient \\
Input & \texttt{const double} & \texttt{epsilon} & Finite Difference Perturbation
\end{tabular}

\vspace{5mm}

\begin{verbatim}
  template <typename F>
  void finite_diff_gradient(const F& f,
                            const Eigen::VectorXd& x,
                            Eigen::VectorXd& g,
                            const double epsilon = 1e-6)
\end{verbatim}

\end{tcolorbox}

\subsection{First-Order Methods}

These methods require a functor implementing
%
\begin{verbatim}
var<1U> operator()(const Eigen::VectorXd& x) const
\end{verbatim}

\begin{tcolorbox}[colback=white,colframe=gray90, coltitle=black,boxrule=3pt,
fonttitle=\bfseries,title=Gradient]

\begin{tabular}{llll}
\textit{I/O} & \textit{Type} & \textit{Parameter} & \textit{Description} \\
Input & \texttt{const F\&} & \texttt{f} & Functor Implementation \\
Input & \texttt{const Eigen::VectorXd\&} & \texttt{x} & Input Values \\
Output & \texttt{Eigen::VectorXd\&} & \texttt{g} & Gradient
\end{tabular}

\vspace{5mm}

\begin{verbatim}
  template <typename F>
  void gradient(const F& f,
                const Eigen::VectorXd& x,
                Eigen::VectorXd& g)
\end{verbatim}

\end{tcolorbox}

\begin{tcolorbox}[colback=white,colframe=gray90, coltitle=black,boxrule=3pt,
fonttitle=\bfseries,title=Gradient]

\begin{tabular}{llll}
\textit{I/O} & \textit{Type} & \textit{Parameter} & \textit{Description} \\
Input & \texttt{const F\&} & \texttt{f} & Functor Implementation \\
Input & \texttt{const Eigen::VectorXd\&} & \texttt{x} & Input Values \\
Output & \texttt{double\&} & \texttt{fx} & Function Value \\
Output & \texttt{Eigen::VectorXd\&} & \texttt{g} & Gradient
\end{tabular}

\vspace{5mm}

\begin{verbatim}
  template <typename F>
  void gradient(const F& f,
                const Eigen::VectorXd& x,
                double& fx,
                Eigen::VectorXd& g)
\end{verbatim}

\end{tcolorbox}

\begin{tcolorbox}[colback=white,colframe=gray90, coltitle=black,boxrule=3pt,
fonttitle=\bfseries,title=Directional Derivative]

\begin{tabular}{llll}
\textit{I/O} & \textit{Type} & \textit{Parameter} & \textit{Description} \\
Input & \texttt{const F\&} & \texttt{f} & Functor Implementation \\
Input & \texttt{const Eigen::VectorXd\&} & \texttt{x} & Input Values \\
Input & \texttt{const Eigen::VectorXd\&} & \texttt{v} & Input Direction \\
Output & \texttt{double\&} & \texttt{g} & Directional Derivative
\end{tabular}

\vspace{5mm}

\begin{verbatim}
  template <typename F>
  void gradient_dot_vector(const F& f,
                           const Eigen::VectorXd& x,
                           const Eigen::VectorXd& v,
                           double& grad_dot_v)
\end{verbatim}

\end{tcolorbox}

\begin{tcolorbox}[colback=white,colframe=gray90, coltitle=black,boxrule=3pt,
fonttitle=\bfseries,title=Directional Derivative]

\begin{tabular}{llll}
\textit{I/O} & \textit{Type} & \textit{Parameter} & \textit{Description} \\
Input & \texttt{const F\&} & \texttt{f} & Functor Implementation \\
Input & \texttt{const Eigen::VectorXd\&} & \texttt{x} & Input Values \\
Input & \texttt{const Eigen::VectorXd\&} & \texttt{v} & Input Direction \\
Output & \texttt{double\&} & \texttt{f} & Function Value \\
Output & \texttt{double\&} & \texttt{g} & Directional Derivative
\end{tabular}

\vspace{5mm}

\begin{verbatim}
  template <typename F>
  void gradient_dot_vector(const F& f,
                           const Eigen::VectorXd& x,
                           const Eigen::VectorXd& v,
                           double& f,
                           double& grad_dot_v)
\end{verbatim}

\end{tcolorbox}

\begin{tcolorbox}[colback=white,colframe=gray90, coltitle=black,boxrule=3pt,
fonttitle=\bfseries,title=Finite Difference Hessian]

\begin{tabular}{llll}
\textit{I/O} & \textit{Type} & \textit{Parameter} & \textit{Description} \\
Input & \texttt{const F\&} & \texttt{f} & Functor Implementation \\
Input & \texttt{const Eigen::VectorXd\&} & \texttt{x} & Input Values \\
Output & \texttt{const Eigen::MatrixXd\&} & \texttt{H} & Hessian \\
Output & \texttt{const double} & \texttt{epsilon} & Finite Difference Perturbation
\end{tabular}

\vspace{5mm}

\begin{verbatim}
  template <typename F>
  void finite_diff_hessian(const F& f,
                           const Eigen::VectorXd& x,
                           Eigen::MatrixXd& H,
                           const double epsilon = 1e-6)
\end{verbatim}

\end{tcolorbox}

\subsection{Second-Order Methods}

These methods require a functor implementing
%
\begin{verbatim}
var<2U> operator()(const Eigen::VectorXd& x) const
\end{verbatim}

\begin{tcolorbox}[colback=white,colframe=gray90, coltitle=black,boxrule=3pt,
fonttitle=\bfseries,title=Hessian]

\begin{tabular}{llll}
\textit{I/O} & \textit{Type} & \textit{Parameter} & \textit{Description} \\
Input & \texttt{const F\&} & \texttt{f} & Functor Implementation \\
Input & \texttt{const Eigen::VectorXd\&} & \texttt{x} & Input Values \\
Output & \texttt{const Eigen::MatrixXd\&} & \texttt{H} & Hessian
\end{tabular}

\vspace{5mm}

\begin{verbatim}
  template <typename F>
  void hessian(const F& f,
               const Eigen::VectorXd& x,
               Eigen::MatrixXd& H)
\end{verbatim}

\end{tcolorbox}

\begin{tcolorbox}[colback=white,colframe=gray90, coltitle=black,boxrule=3pt,
fonttitle=\bfseries,title=Hessian]

\begin{tabular}{llll}
\textit{I/O} & \textit{Type} & \textit{Parameter} & \textit{Description} \\
Input & \texttt{const F\&} & \texttt{f} & Functor Implementation \\
Input & \texttt{const Eigen::VectorXd\&} & \texttt{x} & Input Values \\
Output & \texttt{double\&} & \texttt{f} & Function Value \\
Output & \texttt{double\&} & \texttt{g} & Gradient \\
Output & \texttt{const Eigen::MatrixXd\&} & \texttt{H} & Hessian
\end{tabular}

\vspace{5mm}

\begin{verbatim}
  template <typename F>
  void hessian(const F& f,
               const Eigen::VectorXd& x,
               double& f,
               Eigen::VectorXd& g,
               Eigen::MatrixXd& H)
\end{verbatim}

\end{tcolorbox}

\begin{tcolorbox}[colback=white,colframe=gray90, coltitle=black,boxrule=3pt,
fonttitle=\bfseries,title=Hessian-Vector Product]

\begin{tabular}{llll}
\textit{I/O} & \textit{Type} & \textit{Parameter} & \textit{Description} \\
Input & \texttt{const F\&} & \texttt{f} & Functor Implementation \\
Input & \texttt{const Eigen::VectorXd\&} & \texttt{x} & Input Values \\
Input & \texttt{const Eigen::VectorXd\&} & \texttt{v} & Input Direction \\
Output & \texttt{const Eigen::VectorXd\&} & \texttt{H} & Hessian-Vector Product
\end{tabular}

\vspace{5mm}

\begin{verbatim}
  template <typename F>
  void hessian_dot_vector(const F& f,
                          const Eigen::VectorXd& x,
                          const Eigen::VectorXd& v,
                          Eigen::VectorXd& hessian_dot_v)
\end{verbatim}

\end{tcolorbox}

\begin{tcolorbox}[colback=white,colframe=gray90, coltitle=black,boxrule=3pt,
fonttitle=\bfseries,title=Hessian-Vector Product]

\begin{tabular}{llll}
\textit{I/O} & \textit{Type} & \textit{Parameter} & \textit{Description} \\
Input & \texttt{const F\&} & \texttt{f} & Functor Implementation \\
Input & \texttt{const Eigen::VectorXd\&} & \texttt{x} & Input Values \\
Input & \texttt{const Eigen::VectorXd\&} & \texttt{v} & Input Direction \\
Output & \texttt{double\&} & \texttt{f} & Function Value \\
Output & \texttt{double\&} & \texttt{g} & Gradient \\
Output & \texttt{const Eigen::VectorXd\&} & \texttt{H} & Hessian-Vector Product
\end{tabular}

\vspace{5mm}

\begin{verbatim}
  template <typename F>
  void hessian_dot_vector(const F& f,
                          const Eigen::VectorXd& x,
                          const Eigen::VectorXd& v,
                          double& f,
                          Eigen::VectorXd& g,
                          Eigen::VectorXd& hessian_dot_v)
\end{verbatim}

\end{tcolorbox}

\begin{tcolorbox}[colback=white,colframe=gray90, coltitle=black,boxrule=3pt,
fonttitle=\bfseries,title=Trace of Matrix-Hessian Product]

\begin{tabular}{llll}
\textit{I/O} & \textit{Type} & \textit{Parameter} & \textit{Description} \\
Input & \texttt{const F\&} & \texttt{f} & Functor Implementation \\
Input & \texttt{const Eigen::VectorXd\&} & \texttt{x} & Input Values \\
Input & \texttt{const Eigen::MatrixXd\&} & \texttt{M} & Input Matrix \\
Output & \texttt{double\&} & \texttt{H} & Trace of Matrix-Hessian Product
\end{tabular}

\vspace{5mm}

\begin{verbatim}
  template <typename F>
  void trace_matrix_times_hessian(const F& f,
                                  const Eigen::VectorXd& x,
                                  const Eigen::MatrixXd& M,
                                  double& trace_m_times_h)
\end{verbatim}

\end{tcolorbox}

\begin{tcolorbox}[colback=white,colframe=gray90, coltitle=black,boxrule=3pt,
fonttitle=\bfseries,title=Trace of Matrix-Hessian Product]

\begin{tabular}{llll}
\textit{I/O} & \textit{Type} & \textit{Parameter} & \textit{Description} \\
Input & \texttt{const F\&} & \texttt{f} & Functor Implementation \\
Input & \texttt{const Eigen::VectorXd\&} & \texttt{x} & Input Values \\
Input & \texttt{const Eigen::MatrixXd\&} & \texttt{M} & Input Matrix \\
Output & \texttt{double\&} & \texttt{f} & Function Value \\
Output & \texttt{double\&} & \texttt{g} & Gradient \\
Output & \texttt{double\&} & \texttt{H} & Trace of Matrix-Hessian Product
\end{tabular}

\vspace{5mm}

\begin{verbatim}
  template <typename F>
  void trace_matrix_times_hessian(const F& f,
                                  const Eigen::VectorXd& x,
                                  const Eigen::MatrixXd& M,
                                  double& f,
                                  Eigen::VectorXd& g,
                                  double& trace_m_times_h)
\end{verbatim}

\end{tcolorbox}

\begin{tcolorbox}[colback=white,colframe=gray90, coltitle=black,boxrule=3pt,
fonttitle=\bfseries,title=Finite Difference Hessian Gradient]

\begin{tabular}{llll}
\textit{I/O} & \textit{Type} & \textit{Parameter} & \textit{Description} \\
Input & \texttt{const F\&} & \texttt{f} & Functor Implementation \\
Input & \texttt{const Eigen::VectorXd\&} & \texttt{x} & Input Values \\
Output & \texttt{const Eigen::MatrixXd\&} & \texttt{grad\_H} & Hessian Gradient \\
Output & \texttt{const double} & \texttt{epsilon} & Finite Difference Perturbation
\end{tabular}

\vspace{5mm}

\begin{verbatim}
  template <typename F>
  void finite_diff_grad_hessian(const F& f,
                                const Eigen::VectorXd& x,
                                Eigen::MatrixXd& grad_H,
                                const double epsilon = 1e-6)
\end{verbatim}

\vspace{5mm}

The Hessian gradient is stored in a $N \times N^{2}$ matrix, with the $i$th
$N \times N$ block giving the gradient of the Hessian with respect to the
$i$th input parameter.

\end{tcolorbox}

\subsection{Third-Order Methods}

These methods require a functor implementing
%
\begin{verbatim}
var<3U> operator()(const Eigen::VectorXd& x) const
\end{verbatim}

\begin{tcolorbox}[colback=white,colframe=gray90, coltitle=black,boxrule=3pt,
fonttitle=\bfseries,title=Hessian Gradient]

\begin{tabular}{llll}
\textit{I/O} & \textit{Type} & \textit{Parameter} & \textit{Description} \\
Input & \texttt{const F\&} & \texttt{f} & Functor Implementation \\
Input & \texttt{const Eigen::VectorXd\&} & \texttt{x} & Input Values \\
Output & \texttt{const Eigen::MatrixXd\&} & \texttt{grad\_H} & Hessian Gradient \\
\end{tabular}

\vspace{5mm}

\begin{verbatim}
  template <typename F>
  void grad_hessian(const F& f,
                    const Eigen::VectorXd& x,
                    Eigen::MatrixXd& grad_H)
\end{verbatim}

\vspace{5mm}

The Hessian gradient is stored in a $N \times N^{2}$ matrix, with the $i$th
$N \times N$ block giving the gradient of the Hessian with respect to the
$i$th input parameter.

\end{tcolorbox}

\begin{tcolorbox}[colback=white,colframe=gray90, coltitle=black,boxrule=3pt,
fonttitle=\bfseries,title=Hessian Gradient]

\begin{tabular}{llll}
\textit{I/O} & \textit{Type} & \textit{Parameter} & \textit{Description} \\
Input & \texttt{const F\&} & \texttt{f} & Functor Implementation \\
Input & \texttt{const Eigen::VectorXd\&} & \texttt{x} & Input Values \\
Output & \texttt{double\&} & \texttt{f} & Function Value \\
Output & \texttt{double\&} & \texttt{g} & Gradient \\
Output & \texttt{const Eigen::MatrixXd\&} & \texttt{H} & Hessian \\
Output & \texttt{const Eigen::MatrixXd\&} & \texttt{grad\_H} & Hessian Gradient \\
\end{tabular}

\vspace{5mm}

\begin{verbatim}
  template <typename F>
  void grad_hessian(const F& f,
                    const Eigen::VectorXd& x,
                    double& f,
                    Eigen::VectorXd& g,
                    Eigen::MatrixXd& H,
                    Eigen::MatrixXd& grad_H)
\end{verbatim}

\vspace{5mm}

The Hessian gradient is stored in a $N \times N^{2}$ matrix, with the $i$th
$N \times N$ block giving the gradient of the Hessian with respect to the
$i$th input parameter.

\end{tcolorbox}

\begin{tcolorbox}[colback=white,colframe=gray90, coltitle=black,boxrule=3pt,
fonttitle=\bfseries,title=Gradient of Trace of Matrix-Hessian Product]

\begin{tabular}{llll}
\textit{I/O} & \textit{Type} & \textit{Parameter} & \textit{Description} \\
Input & \texttt{const F\&} & \texttt{f} & Functor Implementation \\
Input & \texttt{const Eigen::VectorXd\&} & \texttt{x} & Input Values \\
Input & \texttt{const Eigen::MatrixXd\&} & \texttt{M} & Input Matrix \\
Output & \texttt{const Eigen::VectorXd\&} & \texttt{grad\_trace\_m\_times\_h} 
& Gradient of Trace of Matrix-Hessian Product
\end{tabular}

\vspace{5mm}

\begin{verbatim}
  template <typename F>
  void grad_trace_matrix_times_hessian(const F& f,
                                       const Eigen::VectorXd& x,
                                       const Eigen::MatrixXd& M,
                                       Eigen::VectorXd& grad_trace_m_times_h)
\end{verbatim}

\end{tcolorbox}

\begin{tcolorbox}[colback=white,colframe=gray90, coltitle=black,boxrule=3pt,
fonttitle=\bfseries,title=Gradient of Trace of Matrix-Hessian Product]

\begin{tabular}{llll}
\textit{I/O} & \textit{Type} & \textit{Parameter} & \textit{Description} \\
Input & \texttt{const F\&} & \texttt{f} & Functor Implementation \\
Input & \texttt{const Eigen::VectorXd\&} & \texttt{x} & Input Values \\
Input & \texttt{const Eigen::MatrixXd\&} & \texttt{M} & Input Matrix \\
Output & \texttt{double\&} & \texttt{f} & Function Value \\
Output & \texttt{double\&} & \texttt{g} & Gradient \\
Output & \texttt{const Eigen::MatrixXd\&} & \texttt{H} & Hessian \\
Output & \texttt{const Eigen::VectorXd\&} & \texttt{grad\_trace\_m\_times\_h} 
& Gradient of Trace of Matrix-Hessian Product
\end{tabular}

\vspace{5mm}

\begin{verbatim}
  template <typename F>
  void grad_trace_matrix_times_hessian(const F& f,
                                       const Eigen::VectorXd& x,
                                       const Eigen::MatrixXd& M,
                                       double& f,
                                       Eigen::VectorXd& g,
                                       Eigen::MatrixXd& H,
                                       Eigen::VectorXd& grad_trace_m_times_h)
\end{verbatim}

\end{tcolorbox}

\section{Implemented Operators and Functions} \label{sec:operators_and_functions}

\subsection{Smooth Functions}
\begin{tcolorbox}[colback=white,colframe=gray90, coltitle=black,boxrule=3pt,
fonttitle=\bfseries,title= Acos]

\begin{verbatim}
double acos(double x)

\end{verbatim}

\begin{verbatim}
template <short AutodiffOrder, bool StrictSmoothness, bool ValidateIO>
var<AutodiffOrder, StrictSmoothness, ValidateIO>
  acos(const var<AutodiffOrder, StrictSmoothness, ValidateIO>& input)

\end{verbatim}

\end{tcolorbox}

\begin{tcolorbox}[colback=white,colframe=gray90, coltitle=black,boxrule=3pt,
fonttitle=\bfseries,title= Acosh]

\begin{verbatim}
double acosh(double x)

\end{verbatim}

\begin{verbatim}
template <short AutodiffOrder, bool StrictSmoothness, bool ValidateIO>
var<AutodiffOrder, StrictSmoothness, ValidateIO>
  acosh(const var<AutodiffOrder, StrictSmoothness, ValidateIO>& input)

\end{verbatim}

\end{tcolorbox}

\begin{tcolorbox}[colback=white,colframe=gray90, coltitle=black,boxrule=3pt,
fonttitle=\bfseries,title= Asin]

\begin{verbatim}
double asin(double x)

\end{verbatim}

\begin{verbatim}
template <short AutodiffOrder, bool StrictSmoothness, bool ValidateIO>
var<AutodiffOrder, StrictSmoothness, ValidateIO>
  asin(const var<AutodiffOrder, StrictSmoothness, ValidateIO>& input)

\end{verbatim}

\end{tcolorbox}

\begin{tcolorbox}[colback=white,colframe=gray90, coltitle=black,boxrule=3pt,
fonttitle=\bfseries,title= Asinh]

\begin{verbatim}
double asinh(double x)

\end{verbatim}

\begin{verbatim}
template <short AutodiffOrder, bool StrictSmoothness, bool ValidateIO>
var<AutodiffOrder, StrictSmoothness, ValidateIO>
  asinh(const var<AutodiffOrder, StrictSmoothness, ValidateIO>& input)

\end{verbatim}

\end{tcolorbox}

\begin{tcolorbox}[colback=white,colframe=gray90, coltitle=black,boxrule=3pt,
fonttitle=\bfseries,title= Atan]

\begin{verbatim}
double atan(double x)

\end{verbatim}

\begin{verbatim}
template <short AutodiffOrder, bool StrictSmoothness, bool ValidateIO>
var<AutodiffOrder, StrictSmoothness, ValidateIO>
  atan(const var<AutodiffOrder, StrictSmoothness, ValidateIO>& input)

\end{verbatim}

\end{tcolorbox}

\begin{tcolorbox}[colback=white,colframe=gray90, coltitle=black,boxrule=3pt,
fonttitle=\bfseries,title= Atan2]

\begin{verbatim}
double atan2(double y, double x)

\end{verbatim}

\begin{verbatim}
template <short AutodiffOrder, bool StrictSmoothness, bool ValidateIO>
var<AutodiffOrder, StrictSmoothness, ValidateIO>
  atan2(const var<AutodiffOrder, StrictSmoothness, ValidateIO>& v1,
        const var<AutodiffOrder, StrictSmoothness, ValidateIO>& v2)

\end{verbatim}

\begin{verbatim}
template <short AutodiffOrder, bool StrictSmoothness, bool ValidateIO>
var<AutodiffOrder, StrictSmoothness, ValidateIO>
  atan2(double y,
        const var<AutodiffOrder, StrictSmoothness, ValidateIO>& v2)

\end{verbatim}

\begin{verbatim}
template <short AutodiffOrder, bool StrictSmoothness, bool ValidateIO>
var<AutodiffOrder, StrictSmoothness, ValidateIO>
  atan2(const var<AutodiffOrder, StrictSmoothness, ValidateIO>& v1,
        double x)

\end{verbatim}

\end{tcolorbox}

\begin{tcolorbox}[colback=white,colframe=gray90, coltitle=black,boxrule=3pt,
fonttitle=\bfseries,title= Atanh]

\begin{verbatim}
double atanh(double x)

\end{verbatim}

\begin{verbatim}
template <short AutodiffOrder, bool StrictSmoothness, bool ValidateIO>
var<AutodiffOrder, StrictSmoothness, ValidateIO>
  atanh(const var<AutodiffOrder, StrictSmoothness, ValidateIO>& input)

\end{verbatim}

\end{tcolorbox}

\begin{tcolorbox}[colback=white,colframe=gray90, coltitle=black,boxrule=3pt,
fonttitle=\bfseries,title= Binary Prod Cubes]

\begin{verbatim}
double binary_prod_cubes(double x, double y)

\end{verbatim}

\begin{verbatim}
template <short AutodiffOrder, bool StrictSmoothness, bool ValidateIO>
var<AutodiffOrder, StrictSmoothness, ValidateIO>
  binary_prod_cubes(const var<AutodiffOrder, StrictSmoothness, ValidateIO>& v1,
                    const var<AutodiffOrder, StrictSmoothness, ValidateIO>& v2)

\end{verbatim}

\end{tcolorbox}

\begin{tcolorbox}[colback=white,colframe=gray90, coltitle=black,boxrule=3pt,
fonttitle=\bfseries,title= Cos]

\begin{verbatim}
double cos(double x)

\end{verbatim}

\begin{verbatim}
template <short AutodiffOrder, bool StrictSmoothness, bool ValidateIO>
var<AutodiffOrder, StrictSmoothness, ValidateIO>
  cos(const var<AutodiffOrder, StrictSmoothness, ValidateIO>& input)

\end{verbatim}

\end{tcolorbox}

\begin{tcolorbox}[colback=white,colframe=gray90, coltitle=black,boxrule=3pt,
fonttitle=\bfseries,title= Cosh]

\begin{verbatim}
double cosh(double x)

\end{verbatim}

\begin{verbatim}
template <short AutodiffOrder, bool StrictSmoothness, bool ValidateIO>
var<AutodiffOrder, StrictSmoothness, ValidateIO>
  cosh(const var<AutodiffOrder, StrictSmoothness, ValidateIO>& input)

\end{verbatim}

\end{tcolorbox}

\begin{tcolorbox}[colback=white,colframe=gray90, coltitle=black,boxrule=3pt,
fonttitle=\bfseries,title= Erf]

\begin{verbatim}
double erf(double x)

\end{verbatim}

\begin{verbatim}
template <short AutodiffOrder, bool StrictSmoothness, bool ValidateIO>
var<AutodiffOrder, StrictSmoothness, ValidateIO>
  erf(const var<AutodiffOrder, StrictSmoothness, ValidateIO>& input)

\end{verbatim}

\end{tcolorbox}

\begin{tcolorbox}[colback=white,colframe=gray90, coltitle=black,boxrule=3pt,
fonttitle=\bfseries,title= Erfc]

\begin{verbatim}
double erfc(double x)

\end{verbatim}

\begin{verbatim}
template <short AutodiffOrder, bool StrictSmoothness, bool ValidateIO>
var<AutodiffOrder, StrictSmoothness, ValidateIO>
  erfc(const var<AutodiffOrder, StrictSmoothness, ValidateIO>& input)

\end{verbatim}

\end{tcolorbox}

\begin{tcolorbox}[colback=white,colframe=gray90, coltitle=black,boxrule=3pt,
fonttitle=\bfseries,title= Exp]

\begin{verbatim}
double exp(double x)

\end{verbatim}

\begin{verbatim}
template <short AutodiffOrder, bool StrictSmoothness, bool ValidateIO>
var<AutodiffOrder, StrictSmoothness, ValidateIO>
  exp(const var<AutodiffOrder, StrictSmoothness, ValidateIO>& input)

\end{verbatim}

\end{tcolorbox}

\begin{tcolorbox}[colback=white,colframe=gray90, coltitle=black,boxrule=3pt,
fonttitle=\bfseries,title= Exp2]

\begin{verbatim}
double exp2(double x)

\end{verbatim}

\begin{verbatim}
template <short AutodiffOrder, bool StrictSmoothness, bool ValidateIO>
var<AutodiffOrder, StrictSmoothness, ValidateIO>
  exp2(const var<AutodiffOrder, StrictSmoothness, ValidateIO>& input)

\end{verbatim}

\end{tcolorbox}

\begin{tcolorbox}[colback=white,colframe=gray90, coltitle=black,boxrule=3pt,
fonttitle=\bfseries,title= Expm1]

\begin{verbatim}
double expm1(double x)

\end{verbatim}

\begin{verbatim}
template <short AutodiffOrder, bool StrictSmoothness, bool ValidateIO>
var<AutodiffOrder, StrictSmoothness, ValidateIO>
  expm1(const var<AutodiffOrder, StrictSmoothness, ValidateIO>& input)

\end{verbatim}

\end{tcolorbox}

\begin{tcolorbox}[colback=white,colframe=gray90, coltitle=black,boxrule=3pt,
fonttitle=\bfseries,title= Fma]

\begin{verbatim}
double fma(double x, double y, double z)

\end{verbatim}

\begin{verbatim}
template <short AutodiffOrder, bool StrictSmoothness, bool ValidateIO>
var<AutodiffOrder, StrictSmoothness, ValidateIO>
  fma(const var<AutodiffOrder, StrictSmoothness, ValidateIO>& v1,
      const var<AutodiffOrder, StrictSmoothness, ValidateIO>& v2,
      const var<AutodiffOrder, StrictSmoothness, ValidateIO>& v3)

\end{verbatim}

\end{tcolorbox}

\begin{tcolorbox}[colback=white,colframe=gray90, coltitle=black,boxrule=3pt,
fonttitle=\bfseries,title= Hypot]

\begin{verbatim}
double hypot(double x, double y)

\end{verbatim}

\begin{verbatim}
template <short AutodiffOrder, bool StrictSmoothness, bool ValidateIO>
var<AutodiffOrder, StrictSmoothness, ValidateIO>
  hypot(const var<AutodiffOrder, StrictSmoothness, ValidateIO>& v1,
        const var<AutodiffOrder, StrictSmoothness, ValidateIO>& v2)

\end{verbatim}

\begin{verbatim}
template <short AutodiffOrder, bool StrictSmoothness, bool ValidateIO>
var<AutodiffOrder, StrictSmoothness, ValidateIO>
  hypot(double x,
        const var<AutodiffOrder, StrictSmoothness, ValidateIO>& v2)

\end{verbatim}

\begin{verbatim}
template <short AutodiffOrder, bool StrictSmoothness, bool ValidateIO>
var<AutodiffOrder, StrictSmoothness, ValidateIO>
  hypot(const var<AutodiffOrder, StrictSmoothness, ValidateIO>& v1,
        double y)

\end{verbatim}

\end{tcolorbox}

\begin{tcolorbox}[colback=white,colframe=gray90, coltitle=black,boxrule=3pt,
fonttitle=\bfseries,title= Inv]

\begin{verbatim}
double inv(double x)

\end{verbatim}

\begin{verbatim}
template <short AutodiffOrder, bool StrictSmoothness, bool ValidateIO>
var<AutodiffOrder, StrictSmoothness, ValidateIO>
  inv(const var<AutodiffOrder, StrictSmoothness, ValidateIO>& input)

\end{verbatim}

\end{tcolorbox}

\begin{tcolorbox}[colback=white,colframe=gray90, coltitle=black,boxrule=3pt,
fonttitle=\bfseries,title= Inv Cloglog]

\begin{verbatim}
double inv_cloglog(double x)

\end{verbatim}

\begin{verbatim}
template <short AutodiffOrder, bool StrictSmoothness, bool ValidateIO>
var<AutodiffOrder, StrictSmoothness, ValidateIO>
  inv_cloglog(const var<AutodiffOrder, StrictSmoothness, ValidateIO>& input)

\end{verbatim}

\end{tcolorbox}

\begin{tcolorbox}[colback=white,colframe=gray90, coltitle=black,boxrule=3pt,
fonttitle=\bfseries,title= Inv Logit]

\begin{verbatim}
double inv_logit(double x)

\end{verbatim}

\begin{verbatim}
template <short AutodiffOrder, bool StrictSmoothness, bool ValidateIO>
var<AutodiffOrder, StrictSmoothness, ValidateIO>
  inv_logit(const var<AutodiffOrder, StrictSmoothness, ValidateIO>& input)

\end{verbatim}

\end{tcolorbox}

\begin{tcolorbox}[colback=white,colframe=gray90, coltitle=black,boxrule=3pt,
fonttitle=\bfseries,title= Inv Sqrt]

\begin{verbatim}
double inv_sqrt(double x)

\end{verbatim}

\begin{verbatim}
template <short AutodiffOrder, bool StrictSmoothness, bool ValidateIO>
var<AutodiffOrder, StrictSmoothness, ValidateIO>
  inv_sqrt(const var<AutodiffOrder, StrictSmoothness, ValidateIO>& input)

\end{verbatim}

\end{tcolorbox}

\begin{tcolorbox}[colback=white,colframe=gray90, coltitle=black,boxrule=3pt,
fonttitle=\bfseries,title= Inv Square]

\begin{verbatim}
double inv_square(double x)

\end{verbatim}

\begin{verbatim}
template <short AutodiffOrder, bool StrictSmoothness, bool ValidateIO>
var<AutodiffOrder, StrictSmoothness, ValidateIO>
  inv_square(const var<AutodiffOrder, StrictSmoothness, ValidateIO>& input)

\end{verbatim}

\end{tcolorbox}

\begin{tcolorbox}[colback=white,colframe=gray90, coltitle=black,boxrule=3pt,
fonttitle=\bfseries,title= Lgamma]

\begin{verbatim}
double lgamma(double x)

\end{verbatim}

\begin{verbatim}
template <short AutodiffOrder, bool StrictSmoothness, bool ValidateIO>
var<AutodiffOrder, StrictSmoothness, ValidateIO>
  lgamma(const var<AutodiffOrder, StrictSmoothness, ValidateIO>& input)

\end{verbatim}

\end{tcolorbox}

\begin{tcolorbox}[colback=white,colframe=gray90, coltitle=black,boxrule=3pt,
fonttitle=\bfseries,title= Log]

\begin{verbatim}
double log(double x)

\end{verbatim}

\begin{verbatim}
template <short AutodiffOrder, bool StrictSmoothness, bool ValidateIO>
var<AutodiffOrder, StrictSmoothness, ValidateIO>
  log(const var<AutodiffOrder, StrictSmoothness, ValidateIO>& input)

\end{verbatim}

\end{tcolorbox}

\begin{tcolorbox}[colback=white,colframe=gray90, coltitle=black,boxrule=3pt,
fonttitle=\bfseries,title= Log10]

\begin{verbatim}
double log10(double x)

\end{verbatim}

\begin{verbatim}
template <short AutodiffOrder, bool StrictSmoothness, bool ValidateIO>
var<AutodiffOrder, StrictSmoothness, ValidateIO>
  log10(const var<AutodiffOrder, StrictSmoothness, ValidateIO>& input)

\end{verbatim}

\end{tcolorbox}

\begin{tcolorbox}[colback=white,colframe=gray90, coltitle=black,boxrule=3pt,
fonttitle=\bfseries,title= Log1p]

\begin{verbatim}
double log1p(double x)

\end{verbatim}

\begin{verbatim}
template <short AutodiffOrder, bool StrictSmoothness, bool ValidateIO>
var<AutodiffOrder, StrictSmoothness, ValidateIO>
  log1p(const var<AutodiffOrder, StrictSmoothness, ValidateIO>& input)

\end{verbatim}

\end{tcolorbox}

\begin{tcolorbox}[colback=white,colframe=gray90, coltitle=black,boxrule=3pt,
fonttitle=\bfseries,title= Log1p Exp]

\begin{verbatim}
double log1p_exp(double x)

\end{verbatim}

\begin{verbatim}
template <short AutodiffOrder, bool StrictSmoothness, bool ValidateIO>
var<AutodiffOrder, StrictSmoothness, ValidateIO>
  log1p_exp(const var<AutodiffOrder, StrictSmoothness, ValidateIO>& input)

\end{verbatim}

\end{tcolorbox}

\begin{tcolorbox}[colback=white,colframe=gray90, coltitle=black,boxrule=3pt,
fonttitle=\bfseries,title= Log2]

\begin{verbatim}
double log2(double x)

\end{verbatim}

\begin{verbatim}
template <short AutodiffOrder, bool StrictSmoothness, bool ValidateIO>
var<AutodiffOrder, StrictSmoothness, ValidateIO>
  log2(const var<AutodiffOrder, StrictSmoothness, ValidateIO>& input)

\end{verbatim}

\end{tcolorbox}

\begin{tcolorbox}[colback=white,colframe=gray90, coltitle=black,boxrule=3pt,
fonttitle=\bfseries,title= Log Diff Exp]

\begin{verbatim}
double log_diff_exp(double x, double y)

\end{verbatim}

\begin{verbatim}
template <short AutodiffOrder, bool StrictSmoothness, bool ValidateIO>
var<AutodiffOrder, StrictSmoothness, ValidateIO>
  log_diff_exp(const var<AutodiffOrder, StrictSmoothness, ValidateIO>& v1,
               const var<AutodiffOrder, StrictSmoothness, ValidateIO>& v2)

\end{verbatim}

\begin{verbatim}
template <short AutodiffOrder, bool StrictSmoothness, bool ValidateIO>
var<AutodiffOrder, StrictSmoothness, ValidateIO>
  log_diff_exp(double x,
               const var<AutodiffOrder, StrictSmoothness, ValidateIO>& v2)

\end{verbatim}

\begin{verbatim}
template <short AutodiffOrder, bool StrictSmoothness, bool ValidateIO>
var<AutodiffOrder, StrictSmoothness, ValidateIO>
  log_diff_exp(const var<AutodiffOrder, StrictSmoothness, ValidateIO>& v1,
               double y)

\end{verbatim}

\end{tcolorbox}

\begin{tcolorbox}[colback=white,colframe=gray90, coltitle=black,boxrule=3pt,
fonttitle=\bfseries,title= Log Sum Exp]

\begin{verbatim}
double log_sum_exp(double x, double y)

\end{verbatim}

\begin{verbatim}
template <short AutodiffOrder, bool StrictSmoothness, bool ValidateIO>
var<AutodiffOrder, StrictSmoothness, ValidateIO>
  log_sum_exp(const var<AutodiffOrder, StrictSmoothness, ValidateIO>& v1,
              const var<AutodiffOrder, StrictSmoothness, ValidateIO>& v2)

\end{verbatim}

\begin{verbatim}
template <short AutodiffOrder, bool StrictSmoothness, bool ValidateIO>
var<AutodiffOrder, StrictSmoothness, ValidateIO>
  log_sum_exp(double x,
              const var<AutodiffOrder, StrictSmoothness, ValidateIO>& v2)

\end{verbatim}

\begin{verbatim}
template <short AutodiffOrder, bool StrictSmoothness, bool ValidateIO>
var<AutodiffOrder, StrictSmoothness, ValidateIO>
  log_sum_exp(const var<AutodiffOrder, StrictSmoothness, ValidateIO>& v1,
              double y)

\end{verbatim}

\end{tcolorbox}

\begin{tcolorbox}[colback=white,colframe=gray90, coltitle=black,boxrule=3pt,
fonttitle=\bfseries,title= Multiply Log]

\begin{verbatim}
double multiply_log(double x, double y)

\end{verbatim}

\begin{verbatim}
template <short AutodiffOrder, bool StrictSmoothness, bool ValidateIO>
var<AutodiffOrder, StrictSmoothness, ValidateIO>
  multiply_log(const var<AutodiffOrder, StrictSmoothness, ValidateIO>& v1,
               const var<AutodiffOrder, StrictSmoothness, ValidateIO>& v2)

\end{verbatim}

\begin{verbatim}
template <short AutodiffOrder, bool StrictSmoothness, bool ValidateIO>
var<AutodiffOrder, StrictSmoothness, ValidateIO>
  multiply_log(double x,
               const var<AutodiffOrder, StrictSmoothness, ValidateIO>& v2)

\end{verbatim}

\begin{verbatim}
template <short AutodiffOrder, bool StrictSmoothness, bool ValidateIO>
var<AutodiffOrder, StrictSmoothness, ValidateIO>
  multiply_log(const var<AutodiffOrder, StrictSmoothness, ValidateIO>& v1,
               double y)

\end{verbatim}

\end{tcolorbox}

\begin{tcolorbox}[colback=white,colframe=gray90, coltitle=black,boxrule=3pt,
fonttitle=\bfseries,title= Phi]

\begin{verbatim}
double Phi(double x)

\end{verbatim}

\begin{verbatim}
template <short AutodiffOrder, bool StrictSmoothness, bool ValidateIO>
var<AutodiffOrder, StrictSmoothness, ValidateIO>
  Phi(const var<AutodiffOrder, StrictSmoothness, ValidateIO>& input)

\end{verbatim}

\end{tcolorbox}

\begin{tcolorbox}[colback=white,colframe=gray90, coltitle=black,boxrule=3pt,
fonttitle=\bfseries,title= Phi Approx]

\begin{verbatim}
double Phi_approx(double x)

\end{verbatim}

\begin{verbatim}
template <short AutodiffOrder, bool StrictSmoothness, bool ValidateIO>
var<AutodiffOrder, StrictSmoothness, ValidateIO>
  Phi_approx(const var<AutodiffOrder, StrictSmoothness, ValidateIO>& input)

\end{verbatim}

\end{tcolorbox}

\begin{tcolorbox}[colback=white,colframe=gray90, coltitle=black,boxrule=3pt,
fonttitle=\bfseries,title= Pow]

\begin{verbatim}
double pow(double x, double y)

\end{verbatim}

\begin{verbatim}
template <short AutodiffOrder, bool StrictSmoothness, bool ValidateIO>
var<AutodiffOrder, StrictSmoothness, ValidateIO>
  pow(const var<AutodiffOrder, StrictSmoothness, ValidateIO>& v1,
      const var<AutodiffOrder, StrictSmoothness, ValidateIO>& v2)

\end{verbatim}

\begin{verbatim}
template <short AutodiffOrder, bool StrictSmoothness, bool ValidateIO>
var<AutodiffOrder, StrictSmoothness, ValidateIO>
  pow(double x,
      const var<AutodiffOrder, StrictSmoothness, ValidateIO>& v2)

\end{verbatim}

\begin{verbatim}
template <short AutodiffOrder, bool StrictSmoothness, bool ValidateIO>
var<AutodiffOrder, StrictSmoothness, ValidateIO>
  pow(const var<AutodiffOrder, StrictSmoothness, ValidateIO>& v1,
      double y)

\end{verbatim}

\end{tcolorbox}

\begin{tcolorbox}[colback=white,colframe=gray90, coltitle=black,boxrule=3pt,
fonttitle=\bfseries,title= Sin]

\begin{verbatim}
double sin(double x)

\end{verbatim}

\begin{verbatim}
template <short AutodiffOrder, bool StrictSmoothness, bool ValidateIO>
var<AutodiffOrder, StrictSmoothness, ValidateIO>
  sin(const var<AutodiffOrder, StrictSmoothness, ValidateIO>& input)

\end{verbatim}

\end{tcolorbox}

\begin{tcolorbox}[colback=white,colframe=gray90, coltitle=black,boxrule=3pt,
fonttitle=\bfseries,title= Sinh]

\begin{verbatim}
double sinh(double x)

\end{verbatim}

\begin{verbatim}
template <short AutodiffOrder, bool StrictSmoothness, bool ValidateIO>
var<AutodiffOrder, StrictSmoothness, ValidateIO>
  sinh(const var<AutodiffOrder, StrictSmoothness, ValidateIO>& input)

\end{verbatim}

\end{tcolorbox}

\begin{tcolorbox}[colback=white,colframe=gray90, coltitle=black,boxrule=3pt,
fonttitle=\bfseries,title= Sqrt]

\begin{verbatim}
double sqrt(double x)

\end{verbatim}

\begin{verbatim}
template <short AutodiffOrder, bool StrictSmoothness, bool ValidateIO>
var<AutodiffOrder, StrictSmoothness, ValidateIO>
  sqrt(const var<AutodiffOrder, StrictSmoothness, ValidateIO>& input)

\end{verbatim}

\end{tcolorbox}

\begin{tcolorbox}[colback=white,colframe=gray90, coltitle=black,boxrule=3pt,
fonttitle=\bfseries,title= Square]

\begin{verbatim}
double square(double input)

\end{verbatim}

\begin{verbatim}
template <short AutodiffOrder, bool StrictSmoothness, bool ValidateIO>
var<AutodiffOrder, StrictSmoothness, ValidateIO>
  square(const var<AutodiffOrder, StrictSmoothness, ValidateIO>& input)

\end{verbatim}

\end{tcolorbox}

\begin{tcolorbox}[colback=white,colframe=gray90, coltitle=black,boxrule=3pt,
fonttitle=\bfseries,title= Tan]

\begin{verbatim}
double tan(double x)

\end{verbatim}

\begin{verbatim}
template <short AutodiffOrder, bool StrictSmoothness, bool ValidateIO>
var<AutodiffOrder, StrictSmoothness, ValidateIO>
  tan(const var<AutodiffOrder, StrictSmoothness, ValidateIO>& input)

\end{verbatim}

\end{tcolorbox}

\begin{tcolorbox}[colback=white,colframe=gray90, coltitle=black,boxrule=3pt,
fonttitle=\bfseries,title= Tanh]

\begin{verbatim}
double tanh(double x)

\end{verbatim}

\begin{verbatim}
template <short AutodiffOrder, bool StrictSmoothness, bool ValidateIO>
var<AutodiffOrder, StrictSmoothness, ValidateIO>
  tanh(const var<AutodiffOrder, StrictSmoothness, ValidateIO>& input)

\end{verbatim}

\end{tcolorbox}

\begin{tcolorbox}[colback=white,colframe=gray90, coltitle=black,boxrule=3pt,
fonttitle=\bfseries,title= Tgamma]

\begin{verbatim}
double tgamma(double x)

\end{verbatim}

\begin{verbatim}
template <short AutodiffOrder, bool StrictSmoothness, bool ValidateIO>
var<AutodiffOrder, StrictSmoothness, ValidateIO>
  tgamma(const var<AutodiffOrder, StrictSmoothness, ValidateIO>& input)

\end{verbatim}

\end{tcolorbox}

\begin{tcolorbox}[colback=white,colframe=gray90, coltitle=black,boxrule=3pt,
fonttitle=\bfseries,title= Trinary Prod Cubes]

\begin{verbatim}
double trinary_prod_cubes(double x, double y, double z)

\end{verbatim}

\begin{verbatim}
template <short AutodiffOrder, bool StrictSmoothness, bool ValidateIO>
var<AutodiffOrder, StrictSmoothness, ValidateIO>
  trinary_prod_cubes(const var<AutodiffOrder, StrictSmoothness, ValidateIO>& v1,
                     const var<AutodiffOrder, StrictSmoothness, ValidateIO>& v2,
                     const var<AutodiffOrder, StrictSmoothness, ValidateIO>& v3)

\end{verbatim}

\end{tcolorbox}



\subsection{Non-Smooth Functions}
\begin{tcolorbox}[colback=white,colframe=gray90, coltitle=black,boxrule=3pt,
fonttitle=\bfseries,title= Cbrt]

\begin{verbatim}
double cbrt(double x)

\end{verbatim}

\begin{verbatim}
template <short AutodiffOrder, bool StrictSmoothness, bool ValidateIO>
var<AutodiffOrder, StrictSmoothness, ValidateIO>
  cbrt(const var<AutodiffOrder, StrictSmoothness, ValidateIO>& input)

\end{verbatim}

\end{tcolorbox}

\begin{tcolorbox}[colback=white,colframe=gray90, coltitle=black,boxrule=3pt,
fonttitle=\bfseries,title= Ceil]

\begin{verbatim}
double ceil(double x)

\end{verbatim}

\begin{verbatim}
template <short AutodiffOrder, bool StrictSmoothness, bool ValidateIO>
typename std::enable_if<!StrictSmoothness, var<AutodiffOrder, StrictSmoothness, ValidateIO> >::type
  ceil(const var<AutodiffOrder, StrictSmoothness, ValidateIO>& input)

\end{verbatim}

\end{tcolorbox}

\begin{tcolorbox}[colback=white,colframe=gray90, coltitle=black,boxrule=3pt,
fonttitle=\bfseries,title= Fabs]

\begin{verbatim}
double fabs(double x)

\end{verbatim}

\begin{verbatim}
template <short AutodiffOrder, bool StrictSmoothness, bool ValidateIO>
typename std::enable_if<!StrictSmoothness, var<AutodiffOrder, StrictSmoothness, ValidateIO> >::type
  fabs(const var<AutodiffOrder, StrictSmoothness, ValidateIO>& input)

\end{verbatim}

\end{tcolorbox}

\begin{tcolorbox}[colback=white,colframe=gray90, coltitle=black,boxrule=3pt,
fonttitle=\bfseries,title= Fdim]

\begin{verbatim}
double fdim(double x, double y)

\end{verbatim}

\begin{verbatim}
template <short AutodiffOrder, bool StrictSmoothness, bool ValidateIO>
typename std::enable_if<!StrictSmoothness, var<AutodiffOrder, StrictSmoothness, ValidateIO> >::type
  fdim(const var<AutodiffOrder, StrictSmoothness, ValidateIO>& v1,
       const var<AutodiffOrder, StrictSmoothness, ValidateIO>& v2)

\end{verbatim}

\begin{verbatim}
template <short AutodiffOrder, bool StrictSmoothness, bool ValidateIO>
typename std::enable_if<!StrictSmoothness, var<AutodiffOrder, StrictSmoothness, ValidateIO> >::type
  fdim(double x,
       const var<AutodiffOrder, StrictSmoothness, ValidateIO>& v2)

\end{verbatim}

\begin{verbatim}
template <short AutodiffOrder, bool StrictSmoothness, bool ValidateIO>
typename std::enable_if<!StrictSmoothness, var<AutodiffOrder, StrictSmoothness, ValidateIO> >::type
  fdim(const var<AutodiffOrder, StrictSmoothness, ValidateIO>& v1,
       double y)

\end{verbatim}

\end{tcolorbox}

\begin{tcolorbox}[colback=white,colframe=gray90, coltitle=black,boxrule=3pt,
fonttitle=\bfseries,title= Floor]

\begin{verbatim}
double floor(double x)

\end{verbatim}

\begin{verbatim}
template <short AutodiffOrder, bool StrictSmoothness, bool ValidateIO>
typename std::enable_if<!StrictSmoothness, var<AutodiffOrder, StrictSmoothness, ValidateIO> >::type
  floor(const var<AutodiffOrder, StrictSmoothness, ValidateIO>& input)

\end{verbatim}

\end{tcolorbox}

\begin{tcolorbox}[colback=white,colframe=gray90, coltitle=black,boxrule=3pt,
fonttitle=\bfseries,title= Fmax]

\begin{verbatim}
double fmax(double x, double y)

\end{verbatim}

\begin{verbatim}
template <short AutodiffOrder, bool StrictSmoothness, bool ValidateIO>
typename std::enable_if<!StrictSmoothness, var<AutodiffOrder, StrictSmoothness, ValidateIO> >::type
  fmax(const var<AutodiffOrder, StrictSmoothness, ValidateIO>& v1,
       const var<AutodiffOrder, StrictSmoothness, ValidateIO>& v2)

\end{verbatim}

\begin{verbatim}
template <short AutodiffOrder, bool StrictSmoothness, bool ValidateIO>
typename std::enable_if<!StrictSmoothness, var<AutodiffOrder, StrictSmoothness, ValidateIO> >::type
  fmax(double x,
       const var<AutodiffOrder, StrictSmoothness, ValidateIO>& v2)

\end{verbatim}

\begin{verbatim}
template <short AutodiffOrder, bool StrictSmoothness, bool ValidateIO>
typename std::enable_if<!StrictSmoothness, var<AutodiffOrder, StrictSmoothness, ValidateIO> >::type
  fmax(const var<AutodiffOrder, StrictSmoothness, ValidateIO>& v1,
       double y)

\end{verbatim}

\end{tcolorbox}

\begin{tcolorbox}[colback=white,colframe=gray90, coltitle=black,boxrule=3pt,
fonttitle=\bfseries,title= Fmin]

\begin{verbatim}
double fmin(double x, double y)

\end{verbatim}

\begin{verbatim}
template <short AutodiffOrder, bool StrictSmoothness, bool ValidateIO>
typename std::enable_if<!StrictSmoothness, var<AutodiffOrder, StrictSmoothness, ValidateIO> >::type
  fmin(const var<AutodiffOrder, StrictSmoothness, ValidateIO>& v1,
       const var<AutodiffOrder, StrictSmoothness, ValidateIO>& v2)

\end{verbatim}

\begin{verbatim}
template <short AutodiffOrder, bool StrictSmoothness, bool ValidateIO>
typename std::enable_if<!StrictSmoothness, var<AutodiffOrder, StrictSmoothness, ValidateIO> >::type
  fmin(double x,
       const var<AutodiffOrder, StrictSmoothness, ValidateIO>& v2)

\end{verbatim}

\begin{verbatim}
template <short AutodiffOrder, bool StrictSmoothness, bool ValidateIO>
typename std::enable_if<!StrictSmoothness, var<AutodiffOrder, StrictSmoothness, ValidateIO> >::type
  fmin(const var<AutodiffOrder, StrictSmoothness, ValidateIO>& v1,
       double y)

\end{verbatim}

\end{tcolorbox}

\begin{tcolorbox}[colback=white,colframe=gray90, coltitle=black,boxrule=3pt,
fonttitle=\bfseries,title= Fmod]

\begin{verbatim}
double fmod(double x, double y)

\end{verbatim}

\begin{verbatim}
template <short AutodiffOrder, bool StrictSmoothness, bool ValidateIO>
typename std::enable_if<!StrictSmoothness, var<AutodiffOrder, StrictSmoothness, ValidateIO> >::type
  fmod(const var<AutodiffOrder, StrictSmoothness, ValidateIO>& v1,
       const var<AutodiffOrder, StrictSmoothness, ValidateIO>& v2)

\end{verbatim}

\begin{verbatim}
template <short AutodiffOrder, bool StrictSmoothness, bool ValidateIO>
typename std::enable_if<!StrictSmoothness, var<AutodiffOrder, StrictSmoothness, ValidateIO> >::type
  fmod(double x,
       const var<AutodiffOrder, StrictSmoothness, ValidateIO>& v2)

\end{verbatim}

\begin{verbatim}
template <short AutodiffOrder, bool StrictSmoothness, bool ValidateIO>
typename std::enable_if<!StrictSmoothness, var<AutodiffOrder, StrictSmoothness, ValidateIO> >::type
  fmod(const var<AutodiffOrder, StrictSmoothness, ValidateIO>& v1,
       double y)

\end{verbatim}

\end{tcolorbox}

\begin{tcolorbox}[colback=white,colframe=gray90, coltitle=black,boxrule=3pt,
fonttitle=\bfseries,title= If Else]

\begin{verbatim}
template <short AutodiffOrder, bool StrictSmoothness, bool ValidateIO>
typename std::enable_if<!StrictSmoothness, var<AutodiffOrder, StrictSmoothness, ValidateIO> >::type
  if_else(bool c,
          const var<AutodiffOrder, StrictSmoothness, ValidateIO>& v_true,
          const var<AutodiffOrder, StrictSmoothness, ValidateIO>& v_false)

\end{verbatim}

\begin{verbatim}
template <short AutodiffOrder, bool StrictSmoothness, bool ValidateIO>
typename std::enable_if<!StrictSmoothness, var<AutodiffOrder, StrictSmoothness, ValidateIO> >::type
if_else(bool c,
        double x_true,
        const var<AutodiffOrder, StrictSmoothness, ValidateIO>& v_false)

\end{verbatim}

\begin{verbatim}
template <short AutodiffOrder, bool StrictSmoothness, bool ValidateIO>
typename std::enable_if<!StrictSmoothness, var<AutodiffOrder, StrictSmoothness, ValidateIO> >::type
if_else(bool c,
        const var<AutodiffOrder, StrictSmoothness, ValidateIO>& v_true,
        double x_false)

\end{verbatim}

\end{tcolorbox}

\begin{tcolorbox}[colback=white,colframe=gray90, coltitle=black,boxrule=3pt,
fonttitle=\bfseries,title= Round]

\begin{verbatim}
double round(double x)

\end{verbatim}

\begin{verbatim}
template <short AutodiffOrder, bool StrictSmoothness, bool ValidateIO>
typename std::enable_if<!StrictSmoothness, var<AutodiffOrder, StrictSmoothness, ValidateIO> >::type
  round(const var<AutodiffOrder, StrictSmoothness, ValidateIO>& input)

\end{verbatim}

\end{tcolorbox}

\begin{tcolorbox}[colback=white,colframe=gray90, coltitle=black,boxrule=3pt,
fonttitle=\bfseries,title= Trunc]

\begin{verbatim}
double trunc(double x)

\end{verbatim}

\begin{verbatim}
template <short AutodiffOrder, bool StrictSmoothness, bool ValidateIO>
typename std::enable_if<!StrictSmoothness, var<AutodiffOrder, StrictSmoothness, ValidateIO> >::type
  trunc(const var<AutodiffOrder, StrictSmoothness, ValidateIO>& input)

\end{verbatim}

\end{tcolorbox}



\subsection{Smooth Operators}
\begin{tcolorbox}[colback=white,colframe=gray90, coltitle=black,boxrule=3pt,
fonttitle=\bfseries,title= Operator Addition]

\begin{verbatim}
template <short AutodiffOrder, bool StrictSmoothness, bool ValidateIO>
var<AutodiffOrder, StrictSmoothness, ValidateIO>
  operator+(const var<AutodiffOrder, StrictSmoothness, ValidateIO>& v1,
            const var<AutodiffOrder, StrictSmoothness, ValidateIO>& v2)

\end{verbatim}

\begin{verbatim}
template <short AutodiffOrder, bool StrictSmoothness, bool ValidateIO>
var<AutodiffOrder, StrictSmoothness, ValidateIO>
  operator+(double x,
            const var<AutodiffOrder, StrictSmoothness, ValidateIO>& v2)

\end{verbatim}

\begin{verbatim}
template <short AutodiffOrder, bool StrictSmoothness, bool ValidateIO>
var<AutodiffOrder, StrictSmoothness, ValidateIO>
  operator+(const var<AutodiffOrder, StrictSmoothness, ValidateIO>& v1,
            double y)

\end{verbatim}

\end{tcolorbox}

\begin{tcolorbox}[colback=white,colframe=gray90, coltitle=black,boxrule=3pt,
fonttitle=\bfseries,title= Operator Addition Assignment]

\begin{verbatim}
template <short AutodiffOrder, bool StrictSmoothness, bool ValidateIO>
var<AutodiffOrder, StrictSmoothness, ValidateIO>&
  operator+=(var<AutodiffOrder, StrictSmoothness, ValidateIO>& v1,
             const var<AutodiffOrder, StrictSmoothness, ValidateIO>& v2)

\end{verbatim}

\begin{verbatim}
template <short AutodiffOrder, bool StrictSmoothness, bool ValidateIO>
var<AutodiffOrder, StrictSmoothness, ValidateIO>&
  operator+=(var<AutodiffOrder, StrictSmoothness, ValidateIO>& v1,
             double y)

\end{verbatim}

\end{tcolorbox}

\begin{tcolorbox}[colback=white,colframe=gray90, coltitle=black,boxrule=3pt,
fonttitle=\bfseries,title= Operator Division]

\begin{verbatim}
template <short AutodiffOrder, bool StrictSmoothness, bool ValidateIO>
var<AutodiffOrder, StrictSmoothness, ValidateIO>
  operator/(const var<AutodiffOrder, StrictSmoothness, ValidateIO>& v1,
            const var<AutodiffOrder, StrictSmoothness, ValidateIO>& v2)

\end{verbatim}

\begin{verbatim}
template <short AutodiffOrder, bool StrictSmoothness, bool ValidateIO>
var<AutodiffOrder, StrictSmoothness, ValidateIO>
  operator/(double x,
            const var<AutodiffOrder, StrictSmoothness, ValidateIO>& v2)

\end{verbatim}

\begin{verbatim}
template <short AutodiffOrder, bool StrictSmoothness, bool ValidateIO>
var<AutodiffOrder, StrictSmoothness, ValidateIO>
  operator/(const var<AutodiffOrder, StrictSmoothness, ValidateIO>& v1,
            double y)

\end{verbatim}

\end{tcolorbox}

\begin{tcolorbox}[colback=white,colframe=gray90, coltitle=black,boxrule=3pt,
fonttitle=\bfseries,title= Operator Division Assignment]

\begin{verbatim}
template <short AutodiffOrder, bool StrictSmoothness, bool ValidateIO>
var<AutodiffOrder, StrictSmoothness, ValidateIO>&
  operator/=(var<AutodiffOrder, StrictSmoothness, ValidateIO>& v1,
             const var<AutodiffOrder, StrictSmoothness, ValidateIO>& v2)

\end{verbatim}

\begin{verbatim}
template <short AutodiffOrder, bool StrictSmoothness, bool ValidateIO>
var<AutodiffOrder, StrictSmoothness, ValidateIO>&
  operator/=(var<AutodiffOrder, StrictSmoothness, ValidateIO>& v1,
             double y)

\end{verbatim}

\end{tcolorbox}

\begin{tcolorbox}[colback=white,colframe=gray90, coltitle=black,boxrule=3pt,
fonttitle=\bfseries,title= Operator Multiplication]

\begin{verbatim}
template <short AutodiffOrder, bool StrictSmoothness, bool ValidateIO>
var<AutodiffOrder, StrictSmoothness, ValidateIO>
  operator*(const var<AutodiffOrder, StrictSmoothness, ValidateIO>& v1,
            const var<AutodiffOrder, StrictSmoothness, ValidateIO>& v2)

\end{verbatim}

\begin{verbatim}
template <short AutodiffOrder, bool StrictSmoothness, bool ValidateIO>
var<AutodiffOrder, StrictSmoothness, ValidateIO>
  operator*(double v1,
            const var<AutodiffOrder, StrictSmoothness, ValidateIO>& v2)

\end{verbatim}

\begin{verbatim}
template <short AutodiffOrder, bool StrictSmoothness, bool ValidateIO>
var<AutodiffOrder, StrictSmoothness, ValidateIO>
  operator*(const var<AutodiffOrder, StrictSmoothness, ValidateIO>& v1,
            double v2)

\end{verbatim}

\end{tcolorbox}

\begin{tcolorbox}[colback=white,colframe=gray90, coltitle=black,boxrule=3pt,
fonttitle=\bfseries,title= Operator Multiplication Assignment]

\begin{verbatim}
template <short AutodiffOrder, bool StrictSmoothness, bool ValidateIO>
var<AutodiffOrder, StrictSmoothness, ValidateIO>&
  operator*=(var<AutodiffOrder, StrictSmoothness, ValidateIO>& v1,
             const var<AutodiffOrder, StrictSmoothness, ValidateIO>& v2)

\end{verbatim}

\begin{verbatim}
template <short AutodiffOrder, bool StrictSmoothness, bool ValidateIO>
var<AutodiffOrder, StrictSmoothness, ValidateIO>&
  operator*=(var<AutodiffOrder, StrictSmoothness, ValidateIO>& v1,
             double v2)

\end{verbatim}

\end{tcolorbox}

\begin{tcolorbox}[colback=white,colframe=gray90, coltitle=black,boxrule=3pt,
fonttitle=\bfseries,title= Operator Subtraction]

\begin{verbatim}
template <short AutodiffOrder, bool StrictSmoothness, bool ValidateIO>
var<AutodiffOrder, StrictSmoothness, ValidateIO>
  operator-(const var<AutodiffOrder, StrictSmoothness, ValidateIO>& v1,
            const var<AutodiffOrder, StrictSmoothness, ValidateIO>& v2)

\end{verbatim}

\begin{verbatim}
template <short AutodiffOrder, bool StrictSmoothness, bool ValidateIO>
var<AutodiffOrder, StrictSmoothness, ValidateIO>
  operator-(double x,
            const var<AutodiffOrder, StrictSmoothness, ValidateIO>& v2)

\end{verbatim}

\begin{verbatim}
template <short AutodiffOrder, bool StrictSmoothness, bool ValidateIO>
var<AutodiffOrder, StrictSmoothness, ValidateIO>
  operator-(const var<AutodiffOrder, StrictSmoothness, ValidateIO>& v1,
            double y)

\end{verbatim}

\end{tcolorbox}

\begin{tcolorbox}[colback=white,colframe=gray90, coltitle=black,boxrule=3pt,
fonttitle=\bfseries,title= Operator Subtraction Assignment]

\begin{verbatim}
template <short AutodiffOrder, bool StrictSmoothness, bool ValidateIO>
var<AutodiffOrder, StrictSmoothness, ValidateIO>&
  operator-=(var<AutodiffOrder, StrictSmoothness, ValidateIO>& v1,
             const var<AutodiffOrder, StrictSmoothness, ValidateIO>& v2)

\end{verbatim}

\begin{verbatim}
template <short AutodiffOrder, bool StrictSmoothness, bool ValidateIO>
var<AutodiffOrder, StrictSmoothness, ValidateIO>&
  operator-=(var<AutodiffOrder, StrictSmoothness, ValidateIO>& v1,
             double y)

\end{verbatim}

\end{tcolorbox}

\begin{tcolorbox}[colback=white,colframe=gray90, coltitle=black,boxrule=3pt,
fonttitle=\bfseries,title= Operator Unary Decrement]

\begin{verbatim}
template <short AutodiffOrder, bool StrictSmoothness, bool ValidateIO>
var<AutodiffOrder, StrictSmoothness, ValidateIO>&
  operator--(var<AutodiffOrder, StrictSmoothness, ValidateIO>& v1)

\end{verbatim}

\begin{verbatim}
template <short AutodiffOrder, bool StrictSmoothness, bool ValidateIO>
var<AutodiffOrder, StrictSmoothness, ValidateIO>
  operator--(const var<AutodiffOrder, StrictSmoothness, ValidateIO>& v1, int /* dummy */)

\end{verbatim}

\end{tcolorbox}

\begin{tcolorbox}[colback=white,colframe=gray90, coltitle=black,boxrule=3pt,
fonttitle=\bfseries,title= Operator Unary Increment]

\begin{verbatim}
template <short AutodiffOrder, bool StrictSmoothness, bool ValidateIO>
var<AutodiffOrder, StrictSmoothness, ValidateIO>&
  operator++(var<AutodiffOrder, StrictSmoothness, ValidateIO>& v1)

\end{verbatim}

\begin{verbatim}
template <short AutodiffOrder, bool StrictSmoothness, bool ValidateIO>
var<AutodiffOrder, StrictSmoothness, ValidateIO>
  operator++(const var<AutodiffOrder, StrictSmoothness, ValidateIO>& v1, int /* dummy */)

\end{verbatim}

\end{tcolorbox}

\begin{tcolorbox}[colback=white,colframe=gray90, coltitle=black,boxrule=3pt,
fonttitle=\bfseries,title= Operator Unary Minus]

\begin{verbatim}
template <short AutodiffOrder, bool StrictSmoothness, bool ValidateIO>
var<AutodiffOrder, StrictSmoothness, ValidateIO>
  operator-(const var<AutodiffOrder, StrictSmoothness, ValidateIO>& v1)

\end{verbatim}

\end{tcolorbox}

\begin{tcolorbox}[colback=white,colframe=gray90, coltitle=black,boxrule=3pt,
fonttitle=\bfseries,title= Operator Unary Plus]

\begin{verbatim}
template <short AutodiffOrder, bool StrictSmoothness, bool ValidateIO>
var<AutodiffOrder, StrictSmoothness, ValidateIO>
  operator+(const var<AutodiffOrder, StrictSmoothness, ValidateIO>& v1)

\end{verbatim}

\end{tcolorbox}



\subsection{Non-Smooth Functions}
\begin{tcolorbox}[colback=white,colframe=gray90, coltitle=black,boxrule=3pt,
fonttitle=\bfseries,title= Operator Equal To]

\begin{verbatim}
template <short AutodiffOrder, bool StrictSmoothness, bool ValidateIO>
typename std::enable_if<!StrictSmoothness, bool >::type
  operator==(const var<AutodiffOrder, StrictSmoothness, ValidateIO>& v1,
             const var<AutodiffOrder, StrictSmoothness, ValidateIO>& v2)

\end{verbatim}

\begin{verbatim}
template <short AutodiffOrder, bool StrictSmoothness, bool ValidateIO>
typename std::enable_if<!StrictSmoothness, bool >::type
  operator==(double x,
             const var<AutodiffOrder, StrictSmoothness, ValidateIO>& v2)

\end{verbatim}

\begin{verbatim}
template <short AutodiffOrder, bool StrictSmoothness, bool ValidateIO>
typename std::enable_if<!StrictSmoothness, bool >::type
  operator==(const var<AutodiffOrder, StrictSmoothness, ValidateIO>& v1,
             double y)

\end{verbatim}

\end{tcolorbox}

\begin{tcolorbox}[colback=white,colframe=gray90, coltitle=black,boxrule=3pt,
fonttitle=\bfseries,title= Operator Greater Than]

\begin{verbatim}
template <short AutodiffOrder, bool StrictSmoothness, bool ValidateIO>
typename std::enable_if<!StrictSmoothness, bool >::type
  operator>(const var<AutodiffOrder, StrictSmoothness, ValidateIO>& v1,
             const var<AutodiffOrder, StrictSmoothness, ValidateIO>& v2)

\end{verbatim}

\begin{verbatim}
template <short AutodiffOrder, bool StrictSmoothness, bool ValidateIO>
typename std::enable_if<!StrictSmoothness, bool >::type
  operator>(double x,
             const var<AutodiffOrder, StrictSmoothness, ValidateIO>& v2)

\end{verbatim}

\begin{verbatim}
template <short AutodiffOrder, bool StrictSmoothness, bool ValidateIO>
typename std::enable_if<!StrictSmoothness, bool >::type
  operator>(const var<AutodiffOrder, StrictSmoothness, ValidateIO>& v1,
             double y)

\end{verbatim}

\end{tcolorbox}

\begin{tcolorbox}[colback=white,colframe=gray90, coltitle=black,boxrule=3pt,
fonttitle=\bfseries,title= Operator Greater Than Or Equal To]

\begin{verbatim}
template <short AutodiffOrder, bool StrictSmoothness, bool ValidateIO>
typename std::enable_if<!StrictSmoothness, bool >::type
  operator>=(const var<AutodiffOrder, StrictSmoothness, ValidateIO>& v1,
             const var<AutodiffOrder, StrictSmoothness, ValidateIO>& v2)

\end{verbatim}

\begin{verbatim}
template <short AutodiffOrder, bool StrictSmoothness, bool ValidateIO>
typename std::enable_if<!StrictSmoothness, bool >::type
  operator>=(double x,
             const var<AutodiffOrder, StrictSmoothness, ValidateIO>& v2)

\end{verbatim}

\begin{verbatim}
template <short AutodiffOrder, bool StrictSmoothness, bool ValidateIO>
typename std::enable_if<!StrictSmoothness, bool >::type
  operator>=(const var<AutodiffOrder, StrictSmoothness, ValidateIO>& v1,
             double y)

\end{verbatim}

\end{tcolorbox}

\begin{tcolorbox}[colback=white,colframe=gray90, coltitle=black,boxrule=3pt,
fonttitle=\bfseries,title= Operator Less Than]

\begin{verbatim}
template <short AutodiffOrder, bool StrictSmoothness, bool ValidateIO>
typename std::enable_if<!StrictSmoothness, bool >::type
  operator<(const var<AutodiffOrder, StrictSmoothness, ValidateIO>& v1,
             const var<AutodiffOrder, StrictSmoothness, ValidateIO>& v2)

\end{verbatim}

\begin{verbatim}
template <short AutodiffOrder, bool StrictSmoothness, bool ValidateIO>
typename std::enable_if<!StrictSmoothness, bool >::type
  operator<(double x,
             const var<AutodiffOrder, StrictSmoothness, ValidateIO>& v2)

\end{verbatim}

\begin{verbatim}
template <short AutodiffOrder, bool StrictSmoothness, bool ValidateIO>
typename std::enable_if<!StrictSmoothness, bool >::type
  operator<(const var<AutodiffOrder, StrictSmoothness, ValidateIO>& v1,
             double y)

\end{verbatim}

\end{tcolorbox}

\begin{tcolorbox}[colback=white,colframe=gray90, coltitle=black,boxrule=3pt,
fonttitle=\bfseries,title= Operator Less Than Or Equal To]

\begin{verbatim}
template <short AutodiffOrder, bool StrictSmoothness, bool ValidateIO>
typename std::enable_if<!StrictSmoothness, bool >::type
  operator<=(const var<AutodiffOrder, StrictSmoothness, ValidateIO>& v1,
             const var<AutodiffOrder, StrictSmoothness, ValidateIO>& v2)

\end{verbatim}

\begin{verbatim}
template <short AutodiffOrder, bool StrictSmoothness, bool ValidateIO>
typename std::enable_if<!StrictSmoothness, bool >::type
  operator<=(double x,
             const var<AutodiffOrder, StrictSmoothness, ValidateIO>& v2)

\end{verbatim}

\begin{verbatim}
template <short AutodiffOrder, bool StrictSmoothness, bool ValidateIO>
typename std::enable_if<!StrictSmoothness, bool >::type
  operator<=(const var<AutodiffOrder, StrictSmoothness, ValidateIO>& v1,
             double y)

\end{verbatim}

\end{tcolorbox}

\begin{tcolorbox}[colback=white,colframe=gray90, coltitle=black,boxrule=3pt,
fonttitle=\bfseries,title= Operator Not Equal To]

\begin{verbatim}
template <short AutodiffOrder, bool StrictSmoothness, bool ValidateIO>
typename std::enable_if<!StrictSmoothness, bool >::type
  operator!=(const var<AutodiffOrder, StrictSmoothness, ValidateIO>& v1,
             const var<AutodiffOrder, StrictSmoothness, ValidateIO>& v2)

\end{verbatim}

\begin{verbatim}
template <short AutodiffOrder, bool StrictSmoothness, bool ValidateIO>
typename std::enable_if<!StrictSmoothness, bool >::type
  operator!=(double x,
             const var<AutodiffOrder, StrictSmoothness, ValidateIO>& v2)

\end{verbatim}

\begin{verbatim}
template <short AutodiffOrder, bool StrictSmoothness, bool ValidateIO>
typename std::enable_if<!StrictSmoothness, bool >::type
  operator!=(const var<AutodiffOrder, StrictSmoothness, ValidateIO>& v1,
             double y)

\end{verbatim}

\end{tcolorbox}

\begin{tcolorbox}[colback=white,colframe=gray90, coltitle=black,boxrule=3pt,
fonttitle=\bfseries,title= Operator Unary Not]

\begin{verbatim}
template <short AutodiffOrder, bool StrictSmoothness, bool ValidateIO>
typename std::enable_if<!StrictSmoothness, bool >::type
  operator!(const var<AutodiffOrder, StrictSmoothness, ValidateIO>& input)

\end{verbatim}

\end{tcolorbox}


