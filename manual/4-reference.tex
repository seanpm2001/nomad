\chapter{Nomad Reference Guide} \label{chap:reference_guide}

\section{Linear Differential Operations} \label{sec:differential_operations}

\subsection{Zeroth-Order Methods}

This methods requires a functor implementing
%
\begin{verbatim}
double operator()(const Eigen::VectorXd& x) const
\end{verbatim}

\begin{tcolorbox}[colback=white,colframe=gray90, coltitle=black,boxrule=3pt,
fonttitle=\bfseries,title=Finite Difference Gradient]

\begin{tabular}{llll}
\textit{I/O} & \textit{Type} & \textit{Parameter} & \textit{Description} \\
Input & \texttt{const F\&} & \texttt{f} & Functor Implementation \\
Input & \texttt{const Eigen::VectorXd\&} & \texttt{x} & Input Values \\
Output & \texttt{Eigen::VectorXd\&} & \texttt{g} & Gradient \\
Input & \texttt{const double} & \texttt{epsilon} & Finite Difference Perturbation
\end{tabular}

\vspace{5mm}

\begin{verbatim}
  template <typename F>
  void finite_diff_gradient(const F& f,
                            const Eigen::VectorXd& x,
                            Eigen::VectorXd& g,
                            const double epsilon = 1e-6)
\end{verbatim}

\end{tcolorbox}

\subsection{First-Order Methods}

These methods require a functor implementing
%
\begin{verbatim}
var<1U> operator()(const Eigen::VectorXd& x) const
\end{verbatim}

\begin{tcolorbox}[colback=white,colframe=gray90, coltitle=black,boxrule=3pt,
fonttitle=\bfseries,title=Gradient]

\begin{tabular}{llll}
\textit{I/O} & \textit{Type} & \textit{Parameter} & \textit{Description} \\
Input & \texttt{const F\&} & \texttt{f} & Functor Implementation \\
Input & \texttt{const Eigen::VectorXd\&} & \texttt{x} & Input Values \\
Output & \texttt{Eigen::VectorXd\&} & \texttt{g} & Gradient
\end{tabular}

\vspace{5mm}

\begin{verbatim}
  template <typename F>
  void gradient(const F& f,
                const Eigen::VectorXd& x,
                Eigen::VectorXd& g)
\end{verbatim}

\end{tcolorbox}

\begin{tcolorbox}[colback=white,colframe=gray90, coltitle=black,boxrule=3pt,
fonttitle=\bfseries,title=Gradient]

\begin{tabular}{llll}
\textit{I/O} & \textit{Type} & \textit{Parameter} & \textit{Description} \\
Input & \texttt{const F\&} & \texttt{f} & Functor Implementation \\
Input & \texttt{const Eigen::VectorXd\&} & \texttt{x} & Input Values \\
Output & \texttt{double\&} & \texttt{fx} & Function Value \\
Output & \texttt{Eigen::VectorXd\&} & \texttt{g} & Gradient
\end{tabular}

\vspace{5mm}

\begin{verbatim}
  template <typename F>
  void gradient(const F& f,
                const Eigen::VectorXd& x,
                double& fx,
                Eigen::VectorXd& g)
\end{verbatim}

\end{tcolorbox}

\begin{tcolorbox}[colback=white,colframe=gray90, coltitle=black,boxrule=3pt,
fonttitle=\bfseries,title=Directional Derivative]

\begin{tabular}{llll}
\textit{I/O} & \textit{Type} & \textit{Parameter} & \textit{Description} \\
Input & \texttt{const F\&} & \texttt{f} & Functor Implementation \\
Input & \texttt{const Eigen::VectorXd\&} & \texttt{x} & Input Values \\
Input & \texttt{const Eigen::VectorXd\&} & \texttt{v} & Input Direction \\
Output & \texttt{double\&} & \texttt{g} & Directional Derivative
\end{tabular}

\vspace{5mm}

\begin{verbatim}
  template <typename F>
  void gradient_dot_vector(const F& f,
                           const Eigen::VectorXd& x,
                           const Eigen::VectorXd& v,
                           double& grad_dot_v)
\end{verbatim}

\end{tcolorbox}

\begin{tcolorbox}[colback=white,colframe=gray90, coltitle=black,boxrule=3pt,
fonttitle=\bfseries,title=Directional Derivative]

\begin{tabular}{llll}
\textit{I/O} & \textit{Type} & \textit{Parameter} & \textit{Description} \\
Input & \texttt{const F\&} & \texttt{f} & Functor Implementation \\
Input & \texttt{const Eigen::VectorXd\&} & \texttt{x} & Input Values \\
Input & \texttt{const Eigen::VectorXd\&} & \texttt{v} & Input Direction \\
Output & \texttt{double\&} & \texttt{f} & Function Value \\
Output & \texttt{double\&} & \texttt{g} & Directional Derivative
\end{tabular}

\vspace{5mm}

\begin{verbatim}
  template <typename F>
  void gradient_dot_vector(const F& f,
                           const Eigen::VectorXd& x,
                           const Eigen::VectorXd& v,
                           double& f,
                           double& grad_dot_v)
\end{verbatim}

\end{tcolorbox}

\begin{tcolorbox}[colback=white,colframe=gray90, coltitle=black,boxrule=3pt,
fonttitle=\bfseries,title=Finite Difference Hessian]

\begin{tabular}{llll}
\textit{I/O} & \textit{Type} & \textit{Parameter} & \textit{Description} \\
Input & \texttt{const F\&} & \texttt{f} & Functor Implementation \\
Input & \texttt{const Eigen::VectorXd\&} & \texttt{x} & Input Values \\
Output & \texttt{const Eigen::MatrixXd\&} & \texttt{H} & Hessian \\
Output & \texttt{const double} & \texttt{epsilon} & Finite Difference Perturbation
\end{tabular}

\vspace{5mm}

\begin{verbatim}
  template <typename F>
  void finite_diff_hessian(const F& f,
                           const Eigen::VectorXd& x,
                           Eigen::MatrixXd& H,
                           const double epsilon = 1e-6)
\end{verbatim}

\end{tcolorbox}

\subsection{Second-Order Methods}

These methods require a functor implementing
%
\begin{verbatim}
var<2U> operator()(const Eigen::VectorXd& x) const
\end{verbatim}

\begin{tcolorbox}[colback=white,colframe=gray90, coltitle=black,boxrule=3pt,
fonttitle=\bfseries,title=Hessian]

\begin{tabular}{llll}
\textit{I/O} & \textit{Type} & \textit{Parameter} & \textit{Description} \\
Input & \texttt{const F\&} & \texttt{f} & Functor Implementation \\
Input & \texttt{const Eigen::VectorXd\&} & \texttt{x} & Input Values \\
Output & \texttt{const Eigen::MatrixXd\&} & \texttt{H} & Hessian
\end{tabular}

\vspace{5mm}

\begin{verbatim}
  template <typename F>
  void hessian(const F& f,
               const Eigen::VectorXd& x,
               Eigen::MatrixXd& H)
\end{verbatim}

\end{tcolorbox}

\begin{tcolorbox}[colback=white,colframe=gray90, coltitle=black,boxrule=3pt,
fonttitle=\bfseries,title=Hessian]

\begin{tabular}{llll}
\textit{I/O} & \textit{Type} & \textit{Parameter} & \textit{Description} \\
Input & \texttt{const F\&} & \texttt{f} & Functor Implementation \\
Input & \texttt{const Eigen::VectorXd\&} & \texttt{x} & Input Values \\
Output & \texttt{double\&} & \texttt{f} & Function Value \\
Output & \texttt{double\&} & \texttt{g} & Gradient \\
Output & \texttt{const Eigen::MatrixXd\&} & \texttt{H} & Hessian
\end{tabular}

\vspace{5mm}

\begin{verbatim}
  template <typename F>
  void hessian(const F& f,
               const Eigen::VectorXd& x,
               double& f,
               Eigen::VectorXd& g,
               Eigen::MatrixXd& H)
\end{verbatim}

\end{tcolorbox}

\begin{tcolorbox}[colback=white,colframe=gray90, coltitle=black,boxrule=3pt,
fonttitle=\bfseries,title=Hessian-Vector Product]

\begin{tabular}{llll}
\textit{I/O} & \textit{Type} & \textit{Parameter} & \textit{Description} \\
Input & \texttt{const F\&} & \texttt{f} & Functor Implementation \\
Input & \texttt{const Eigen::VectorXd\&} & \texttt{x} & Input Values \\
Input & \texttt{const Eigen::VectorXd\&} & \texttt{v} & Input Direction \\
Output & \texttt{const Eigen::VectorXd\&} & \texttt{H} & Hessian-Vector Product
\end{tabular}

\vspace{5mm}

\begin{verbatim}
  template <typename F>
  void hessian_dot_vector(const F& f,
                          const Eigen::VectorXd& x,
                          const Eigen::VectorXd& v,
                          Eigen::VectorXd& hessian_dot_v)
\end{verbatim}

\end{tcolorbox}

\begin{tcolorbox}[colback=white,colframe=gray90, coltitle=black,boxrule=3pt,
fonttitle=\bfseries,title=Hessian-Vector Product]

\begin{tabular}{llll}
\textit{I/O} & \textit{Type} & \textit{Parameter} & \textit{Description} \\
Input & \texttt{const F\&} & \texttt{f} & Functor Implementation \\
Input & \texttt{const Eigen::VectorXd\&} & \texttt{x} & Input Values \\
Input & \texttt{const Eigen::VectorXd\&} & \texttt{v} & Input Direction \\
Output & \texttt{double\&} & \texttt{f} & Function Value \\
Output & \texttt{double\&} & \texttt{g} & Gradient \\
Output & \texttt{const Eigen::VectorXd\&} & \texttt{H} & Hessian-Vector Product
\end{tabular}

\vspace{5mm}

\begin{verbatim}
  template <typename F>
  void hessian_dot_vector(const F& f,
                          const Eigen::VectorXd& x,
                          const Eigen::VectorXd& v,
                          double& f,
                          Eigen::VectorXd& g,
                          Eigen::VectorXd& hessian_dot_v)
\end{verbatim}

\end{tcolorbox}

\begin{tcolorbox}[colback=white,colframe=gray90, coltitle=black,boxrule=3pt,
fonttitle=\bfseries,title=Trace of Matrix-Hessian Product]

\begin{tabular}{llll}
\textit{I/O} & \textit{Type} & \textit{Parameter} & \textit{Description} \\
Input & \texttt{const F\&} & \texttt{f} & Functor Implementation \\
Input & \texttt{const Eigen::VectorXd\&} & \texttt{x} & Input Values \\
Input & \texttt{const Eigen::MatrixXd\&} & \texttt{M} & Input Matrix \\
Output & \texttt{double\&} & \texttt{H} & Trace of Matrix-Hessian Product
\end{tabular}

\vspace{5mm}

\begin{verbatim}
  template <typename F>
  void trace_matrix_times_hessian(const F& f,
                                  const Eigen::VectorXd& x,
                                  const Eigen::MatrixXd& M,
                                  double& trace_m_times_h)
\end{verbatim}

\end{tcolorbox}

\begin{tcolorbox}[colback=white,colframe=gray90, coltitle=black,boxrule=3pt,
fonttitle=\bfseries,title=Trace of Matrix-Hessian Product]

\begin{tabular}{llll}
\textit{I/O} & \textit{Type} & \textit{Parameter} & \textit{Description} \\
Input & \texttt{const F\&} & \texttt{f} & Functor Implementation \\
Input & \texttt{const Eigen::VectorXd\&} & \texttt{x} & Input Values \\
Input & \texttt{const Eigen::MatrixXd\&} & \texttt{M} & Input Matrix \\
Output & \texttt{double\&} & \texttt{f} & Function Value \\
Output & \texttt{double\&} & \texttt{g} & Gradient \\
Output & \texttt{double\&} & \texttt{H} & Trace of Matrix-Hessian Product
\end{tabular}

\vspace{5mm}

\begin{verbatim}
  template <typename F>
  void trace_matrix_times_hessian(const F& f,
                                  const Eigen::VectorXd& x,
                                  const Eigen::MatrixXd& M,
                                  double& f,
                                  Eigen::VectorXd& g,
                                  double& trace_m_times_h)
\end{verbatim}

\end{tcolorbox}

\begin{tcolorbox}[colback=white,colframe=gray90, coltitle=black,boxrule=3pt,
fonttitle=\bfseries,title=Finite Difference Hessian Gradient]

\begin{tabular}{llll}
\textit{I/O} & \textit{Type} & \textit{Parameter} & \textit{Description} \\
Input & \texttt{const F\&} & \texttt{f} & Functor Implementation \\
Input & \texttt{const Eigen::VectorXd\&} & \texttt{x} & Input Values \\
Output & \texttt{const Eigen::MatrixXd\&} & \texttt{grad\_H} & Hessian Gradient \\
Output & \texttt{const double} & \texttt{epsilon} & Finite Difference Perturbation
\end{tabular}

\vspace{5mm}

\begin{verbatim}
  template <typename F>
  void finite_diff_grad_hessian(const F& f,
                                const Eigen::VectorXd& x,
                                Eigen::MatrixXd& grad_H,
                                const double epsilon = 1e-6)
\end{verbatim}

\vspace{5mm}

The Hessian gradient is stored in a $N \times N^{2}$ matrix, with the $i$th
$N \times N$ block giving the gradient of the Hessian with respect to the
$i$th input parameter.

\end{tcolorbox}

\subsection{Third-Order Methods}

These methods require a functor implementing
%
\begin{verbatim}
var<3U> operator()(const Eigen::VectorXd& x) const
\end{verbatim}

\begin{tcolorbox}[colback=white,colframe=gray90, coltitle=black,boxrule=3pt,
fonttitle=\bfseries,title=Hessian Gradient]

\begin{tabular}{llll}
\textit{I/O} & \textit{Type} & \textit{Parameter} & \textit{Description} \\
Input & \texttt{const F\&} & \texttt{f} & Functor Implementation \\
Input & \texttt{const Eigen::VectorXd\&} & \texttt{x} & Input Values \\
Output & \texttt{const Eigen::MatrixXd\&} & \texttt{grad\_H} & Hessian Gradient \\
\end{tabular}

\vspace{5mm}

\begin{verbatim}
  template <typename F>
  void grad_hessian(const F& f,
                    const Eigen::VectorXd& x,
                    Eigen::MatrixXd& grad_H)
\end{verbatim}

\vspace{5mm}

The Hessian gradient is stored in a $N \times N^{2}$ matrix, with the $i$th
$N \times N$ block giving the gradient of the Hessian with respect to the
$i$th input parameter.

\end{tcolorbox}

\begin{tcolorbox}[colback=white,colframe=gray90, coltitle=black,boxrule=3pt,
fonttitle=\bfseries,title=Hessian Gradient]

\begin{tabular}{llll}
\textit{I/O} & \textit{Type} & \textit{Parameter} & \textit{Description} \\
Input & \texttt{const F\&} & \texttt{f} & Functor Implementation \\
Input & \texttt{const Eigen::VectorXd\&} & \texttt{x} & Input Values \\
Output & \texttt{double\&} & \texttt{f} & Function Value \\
Output & \texttt{double\&} & \texttt{g} & Gradient \\
Output & \texttt{const Eigen::MatrixXd\&} & \texttt{H} & Hessian \\
Output & \texttt{const Eigen::MatrixXd\&} & \texttt{grad\_H} & Hessian Gradient \\
\end{tabular}

\vspace{5mm}

\begin{verbatim}
  template <typename F>
  void grad_hessian(const F& f,
                    const Eigen::VectorXd& x,
                    double& f,
                    Eigen::VectorXd& g,
                    Eigen::MatrixXd& H,
                    Eigen::MatrixXd& grad_H)
\end{verbatim}

\vspace{5mm}

The Hessian gradient is stored in a $N \times N^{2}$ matrix, with the $i$th
$N \times N$ block giving the gradient of the Hessian with respect to the
$i$th input parameter.

\end{tcolorbox}

\begin{tcolorbox}[colback=white,colframe=gray90, coltitle=black,boxrule=3pt,
fonttitle=\bfseries,title=Gradient of Trace of Matrix-Hessian Product]

\begin{tabular}{llll}
\textit{I/O} & \textit{Type} & \textit{Parameter} & \textit{Description} \\
Input & \texttt{const F\&} & \texttt{f} & Functor Implementation \\
Input & \texttt{const Eigen::VectorXd\&} & \texttt{x} & Input Values \\
Input & \texttt{const Eigen::MatrixXd\&} & \texttt{M} & Input Matrix \\
Output & \texttt{const Eigen::VectorXd\&} & \texttt{grad\_trace\_m\_times\_h} 
& Gradient of Trace of Matrix-Hessian Product
\end{tabular}

\vspace{5mm}

\begin{verbatim}
  template <typename F>
  void grad_trace_matrix_times_hessian(const F& f,
                                       const Eigen::VectorXd& x,
                                       const Eigen::MatrixXd& M,
                                       Eigen::VectorXd& grad_trace_m_times_h)
\end{verbatim}

\end{tcolorbox}

\begin{tcolorbox}[colback=white,colframe=gray90, coltitle=black,boxrule=3pt,
fonttitle=\bfseries,title=Gradient of Trace of Matrix-Hessian Product]

\begin{tabular}{llll}
\textit{I/O} & \textit{Type} & \textit{Parameter} & \textit{Description} \\
Input & \texttt{const F\&} & \texttt{f} & Functor Implementation \\
Input & \texttt{const Eigen::VectorXd\&} & \texttt{x} & Input Values \\
Input & \texttt{const Eigen::MatrixXd\&} & \texttt{M} & Input Matrix \\
Output & \texttt{double\&} & \texttt{f} & Function Value \\
Output & \texttt{double\&} & \texttt{g} & Gradient \\
Output & \texttt{const Eigen::MatrixXd\&} & \texttt{H} & Hessian \\
Output & \texttt{const Eigen::VectorXd\&} & \texttt{grad\_trace\_m\_times\_h} 
& Gradient of Trace of Matrix-Hessian Product
\end{tabular}

\vspace{5mm}

\begin{verbatim}
  template <typename F>
  void grad_trace_matrix_times_hessian(const F& f,
                                       const Eigen::VectorXd& x,
                                       const Eigen::MatrixXd& M,
                                       double& f,
                                       Eigen::VectorXd& g,
                                       Eigen::MatrixXd& H,
                                       Eigen::VectorXd& grad_trace_m_times_h)
\end{verbatim}

\end{tcolorbox}

\section{Implemented Operators and Functions} \label{sec:operators_and_functions}

\subsection{Operators}

\begin{tcolorbox}[colback=white,colframe=gray90, coltitle=black,boxrule=3pt,
fonttitle=\bfseries,title=Operator Addition Assignment]

\begin{verbatim}
  template <short autodiff_order>
  inline var<autodiff_order>& operator+=(var<autodiff_order>& v1,
                                         const var<autodiff_order>& v2)
\end{verbatim}

\begin{verbatim}
  template <short autodiff_order>
  inline var<autodiff_order>& operator+=(var<autodiff_order>& v1,
                                         double y)
\end{verbatim}

\end{tcolorbox}

\begin{tcolorbox}[colback=white,colframe=gray90, coltitle=black,boxrule=3pt,
fonttitle=\bfseries,title=Operator Addition]

\begin{verbatim}
  template <short autodiff_order>
  inline var<autodiff_order> operator+(const var<autodiff_order>& v1,
                                       const var<autodiff_order>& v2)
\end{verbatim}

\begin{verbatim}
  template <short autodiff_order>
  inline var<autodiff_order> operator+(double x,
                                       const var<autodiff_order>& v2)
\end{verbatim}

\begin{verbatim}
  template <short autodiff_order>
  inline var<autodiff_order> operator+(const var<autodiff_order>& v1,
                                       double y)
\end{verbatim}

\end{tcolorbox}

\begin{tcolorbox}[colback=white,colframe=gray90, coltitle=black,boxrule=3pt,
fonttitle=\bfseries,title=Operator Assignment]

\begin{verbatim}
    template <short autodiff_order>
    var<autodiff_order>& operator=(const var<autodiff_order>& v)
\end{verbatim}

\begin{verbatim}
    template <short autodiff_order>
    var<autodiff_order>& operator=(double val)
\end{verbatim}

\end{tcolorbox}

\begin{tcolorbox}[colback=white,colframe=gray90, coltitle=black,boxrule=3pt,
fonttitle=\bfseries,title=Operator Division Assignment]

\begin{verbatim}
  template <short autodiff_order>
  inline var<autodiff_order>& operator/=(var<autodiff_order>& v1,
                                         const var<autodiff_order>& v2)
\end{verbatim}

\begin{verbatim}
  template <short autodiff_order>
  inline var<autodiff_order>& operator/=(var<autodiff_order>& v1,
                                         double y)
\end{verbatim}

\end{tcolorbox}

\begin{tcolorbox}[colback=white,colframe=gray90, coltitle=black,boxrule=3pt,
fonttitle=\bfseries,title=Operator Division]

\begin{verbatim}
  template <short autodiff_order>
  inline var<autodiff_order> operator/(const var<autodiff_order>& v1,
                                       const var<autodiff_order>& v2)
\end{verbatim}

\begin{verbatim}
  template <short autodiff_order>
  inline var<autodiff_order> operator/(double x,
                                       const var<autodiff_order>& v2)
\end{verbatim}

\begin{verbatim}
  template <short autodiff_order>
  inline var<autodiff_order> operator/(const var<autodiff_order>& v1,
                                       double y)
\end{verbatim}

\end{tcolorbox}

\begin{tcolorbox}[colback=white,colframe=gray90, coltitle=black,boxrule=3pt,
fonttitle=\bfseries,title=Operator Multiplication]

\begin{verbatim}
  template <short autodiff_order>
  inline var<autodiff_order>& operator*=(var<autodiff_order>& v1,
                                         const var<autodiff_order>& v2)
\end{verbatim}

\begin{verbatim}
  template <short autodiff_order>
  inline var<autodiff_order>& operator*=(var<autodiff_order>& v1,
                                         double v2)
\end{verbatim}

\end{tcolorbox}

\begin{tcolorbox}[colback=white,colframe=gray90, coltitle=black,boxrule=3pt,
fonttitle=\bfseries,title=Operator Multiplication]

\begin{verbatim}
  template <short autodiff_order>
  inline var<autodiff_order> operator*(const var<autodiff_order>& v1,
                                       const var<autodiff_order>& v2)
\end{verbatim}

\begin{verbatim}
  template <short autodiff_order>
  inline var<autodiff_order> operator*(double v1,
                                       const var<autodiff_order>& v2)
\end{verbatim}

\begin{verbatim}
  template <short autodiff_order>
  inline var<autodiff_order> operator*(const var<autodiff_order>& v1,
                                       double v2)
\end{verbatim}

\end{tcolorbox}

\begin{tcolorbox}[colback=white,colframe=gray90, coltitle=black,boxrule=3pt,
fonttitle=\bfseries,title=Operator Subtraction Assignment]

\begin{verbatim}
  template <short autodiff_order>
  inline var<autodiff_order>& operator-=(var<autodiff_order>& v1,
                                         const var<autodiff_order>& v2)
\end{verbatim}

\begin{verbatim}
  template <short autodiff_order>
  inline var<autodiff_order>& operator-=(var<autodiff_order>& v1,
                                         double y)
\end{verbatim}

\end{tcolorbox}

\begin{tcolorbox}[colback=white,colframe=gray90, coltitle=black,boxrule=3pt,
fonttitle=\bfseries,title=Operator Subtraction]

\begin{verbatim}
  template <short autodiff_order>
  inline var<autodiff_order> operator-(const var<autodiff_order>& v1,
                                       const var<autodiff_order>& v2)
\end{verbatim}

\begin{verbatim}
  template <short autodiff_order>
  inline var<autodiff_order> operator-(double x,
                                       const var<autodiff_order>& v2)
\end{verbatim}

\begin{verbatim}
  inline var<autodiff_order> operator-(const var<autodiff_order>& v1,
                                       double y)
\end{verbatim}

\end{tcolorbox}

\begin{tcolorbox}[colback=white,colframe=gray90, coltitle=black,boxrule=3pt,
fonttitle=\bfseries,title=Operator Unary Decrement]

\begin{verbatim}
  template <short autodiff_order>
  inline var<autodiff_order>& operator--(var<autodiff_order>& v1)
\end{verbatim}

\begin{verbatim}
  template <short autodiff_order>
  inline var<autodiff_order> operator--(const var<autodiff_order>& v1, int /* dummy */)
\end{verbatim}

\end{tcolorbox}

\begin{tcolorbox}[colback=white,colframe=gray90, coltitle=black,boxrule=3pt,
fonttitle=\bfseries,title=Operator Unary Increment]

\begin{verbatim}
  template <short autodiff_order>
  inline var<autodiff_order>& operator++(var<autodiff_order>& v1)
\end{verbatim}

\begin{verbatim}
  template <short autodiff_order>
  inline var<autodiff_order> operator++(const var<autodiff_order>& v1, int /* dummy */)
\end{verbatim}

\end{tcolorbox}

\begin{tcolorbox}[colback=white,colframe=gray90, coltitle=black,boxrule=3pt,
fonttitle=\bfseries,title=Operator Unary Minus]

\begin{verbatim}
  template <short autodiff_order>
  inline var<autodiff_order> operator-(const var<autodiff_order>& v1)
\end{verbatim}

\end{tcolorbox}

\begin{tcolorbox}[colback=white,colframe=gray90, coltitle=black,boxrule=3pt,
fonttitle=\bfseries,title=Operator Unary Plus]

\begin{verbatim}
  template <short autodiff_order>
  inline var<autodiff_order> operator+(const var<autodiff_order>& v1)
\end{verbatim}

\end{tcolorbox}

\subsection{Trigonometric and Hyperbolic Functions}

\begin{tcolorbox}[colback=white,colframe=gray90, coltitle=black,boxrule=3pt,
fonttitle=\bfseries,title=Arc Cosine]

\begin{verbatim}
  template <short autodiff_order>
  inline var<autodiff_order> acos(const var<autodiff_order>& input)
\end{verbatim}

\end{tcolorbox}

\begin{tcolorbox}[colback=white,colframe=gray90, coltitle=black,boxrule=3pt,
fonttitle=\bfseries,title=Arc Hyperbolic Cosine]

\begin{verbatim}
  template <short autodiff_order>
  inline var<autodiff_order> acosh(const var<autodiff_order>& input)
\end{verbatim}

\end{tcolorbox}

\begin{tcolorbox}[colback=white,colframe=gray90, coltitle=black,boxrule=3pt,
fonttitle=\bfseries,title=Arc Sine]

\begin{verbatim}
  template <short autodiff_order>
  inline var<autodiff_order> asin(const var<autodiff_order>& input)
\end{verbatim}

\end{tcolorbox}

\begin{tcolorbox}[colback=white,colframe=gray90, coltitle=black,boxrule=3pt,
fonttitle=\bfseries,title=Arc Hyperbolic Sine]

\begin{verbatim}
  template <short autodiff_order>
  inline var<autodiff_order> asinh(const var<autodiff_order>& input)
\end{verbatim}

\end{tcolorbox}

\begin{tcolorbox}[colback=white,colframe=gray90, coltitle=black,boxrule=3pt,
fonttitle=\bfseries,title=Arc Tangent]

\begin{verbatim}
  template <short autodiff_order>
  inline var<autodiff_order> atan(const var<autodiff_order>& input)
\end{verbatim}

\end{tcolorbox}

\begin{tcolorbox}[colback=white,colframe=gray90, coltitle=black,boxrule=3pt,
fonttitle=\bfseries,title=Arc Tangent (Two Arguments)]

\begin{verbatim}
  template <short autodiff_order>
  inline var<autodiff_order> atan2(const var<autodiff_order>& v1,
                                   const var<autodiff_order>& v2)
\end{verbatim}

\begin{verbatim}
  template <short autodiff_order>
  inline var<autodiff_order> atan2(double y,
                                   const var<autodiff_order>& v2)
\end{verbatim}

\begin{verbatim}
  template <short autodiff_order>
  inline var<autodiff_order> atan2(const var<autodiff_order>& v1,
                                   double x)
\end{verbatim}

\end{tcolorbox}

\begin{tcolorbox}[colback=white,colframe=gray90, coltitle=black,boxrule=3pt,
fonttitle=\bfseries,title=Arc Hyperbolic Tangent]

\begin{verbatim}
  template <short autodiff_order>
  inline var<autodiff_order> atanh(const var<autodiff_order>& input)
\end{verbatim}

\end{tcolorbox}

\begin{tcolorbox}[colback=white,colframe=gray90, coltitle=black,boxrule=3pt,
fonttitle=\bfseries,title=Cosine]

\begin{verbatim}
  template <short autodiff_order>
  inline var<autodiff_order> cos(const var<autodiff_order>& input)
\end{verbatim}

\end{tcolorbox}

\begin{tcolorbox}[colback=white,colframe=gray90, coltitle=black,boxrule=3pt,
fonttitle=\bfseries,title=Hyperbolic Cosine]

\begin{verbatim}
  template <short autodiff_order>
  inline var<autodiff_order> cosh(const var<autodiff_order>& input)
\end{verbatim}

\end{tcolorbox}

\begin{tcolorbox}[colback=white,colframe=gray90, coltitle=black,boxrule=3pt,
fonttitle=\bfseries,title=Sine]

\begin{verbatim}
  template <short autodiff_order>
  inline var<autodiff_order> sin(const var<autodiff_order>& input)
\end{verbatim}

\end{tcolorbox}

\begin{tcolorbox}[colback=white,colframe=gray90, coltitle=black,boxrule=3pt,
fonttitle=\bfseries,title=Hyperbolic Sine]

\begin{verbatim}
  template <short autodiff_order>
  inline var<autodiff_order> sinh(const var<autodiff_order>& input)
\end{verbatim}

\end{tcolorbox}

\begin{tcolorbox}[colback=white,colframe=gray90, coltitle=black,boxrule=3pt,
fonttitle=\bfseries,title=Tangent]

\begin{verbatim}
  template <short autodiff_order>
  inline var<autodiff_order> tan(const var<autodiff_order>& input)
\end{verbatim}

\end{tcolorbox}

\begin{tcolorbox}[colback=white,colframe=gray90, coltitle=black,boxrule=3pt,
fonttitle=\bfseries,title=Hyperbolic Tangent]

\begin{verbatim}
  template <short autodiff_order>
  inline var<autodiff_order> tanh(const var<autodiff_order>& input)
\end{verbatim}

\end{tcolorbox}

\subsection{Exponential, Logarithmic, and Logistic Functions}

\begin{tcolorbox}[colback=white,colframe=gray90, coltitle=black,boxrule=3pt,
fonttitle=\bfseries,title=Natural Exponential]

\begin{verbatim}
  template <short autodiff_order>
  inline var<autodiff_order> exp(const var<autodiff_order>& input)
\end{verbatim}

\end{tcolorbox}

\begin{tcolorbox}[colback=white,colframe=gray90, coltitle=black,boxrule=3pt,
fonttitle=\bfseries,title=Binary Exponential]

\begin{verbatim}
  template <short autodiff_order>
  inline var<autodiff_order> exp2(const var<autodiff_order>& input)
\end{verbatim}

\end{tcolorbox}

\begin{tcolorbox}[colback=white,colframe=gray90, coltitle=black,boxrule=3pt,
fonttitle=\bfseries,title=Natural Exponential Minus One]

\begin{verbatim}
  template <short autodiff_order>
  inline var<autodiff_order> expm1(const var<autodiff_order>& input)
\end{verbatim}

\end{tcolorbox}

\begin{tcolorbox}[colback=white,colframe=gray90, coltitle=black,boxrule=3pt,
fonttitle=\bfseries,title=Inverse Complementary Log-Log Link Function]

\begin{verbatim}
  template <short autodiff_order>
  inline var<autodiff_order> inv_cloglog(const var<autodiff_order>& input)
\end{verbatim}

\end{tcolorbox}

\begin{tcolorbox}[colback=white,colframe=gray90, coltitle=black,boxrule=3pt,
fonttitle=\bfseries,title=Inverse Logit or Logistic Function]

\begin{verbatim}
  template <short autodiff_order>
  inline var<autodiff_order> inv_logit(const var<autodiff_order>& input) 
\end{verbatim}

\end{tcolorbox}

\begin{tcolorbox}[colback=white,colframe=gray90, coltitle=black,boxrule=3pt,
fonttitle=\bfseries,title=Natural Logarithm of the Difference of Two Natural Exponentials]

\begin{verbatim}
  template <short autodiff_order>
  inline var<autodiff_order> log_diff_exp(const var<autodiff_order>& v1,
                                          const var<autodiff_order>& v2)
\end{verbatim}

\begin{verbatim}
  template <short autodiff_order>
  inline var<autodiff_order> log_diff_exp(double x,
                                          const var<autodiff_order>& v2)
\end{verbatim}

\begin{verbatim}
  template <short autodiff_order>
  inline var<autodiff_order> log_diff_exp(const var<autodiff_order>& v1,
                                          double y)
\end{verbatim}

\end{tcolorbox}

\begin{tcolorbox}[colback=white,colframe=gray90, coltitle=black,boxrule=3pt,
fonttitle=\bfseries,title=Natural Logarithm of the Sum of Two Natural Exponentials]

\begin{verbatim}
  template <short autodiff_order>
  inline var<autodiff_order> log_sum_exp(const var<autodiff_order>& v1,
                                         const var<autodiff_order>& v2)
\end{verbatim}

\begin{verbatim}
  template <short autodiff_order>
  inline var<autodiff_order> log_sum_exp(double x,
                                         const var<autodiff_order>& v2)
\end{verbatim}

\begin{verbatim}
  template <short autodiff_order>
  inline var<autodiff_order> log_sum_exp(const var<autodiff_order>& v1,
                                         double y)
\end{verbatim}

\end{tcolorbox}

\begin{tcolorbox}[colback=white,colframe=gray90, coltitle=black,boxrule=3pt,
fonttitle=\bfseries,title=Natural Logarithm]

\begin{verbatim}
  template <short autodiff_order>
  inline var<autodiff_order> log(const var<autodiff_order>& input)
\end{verbatim}

\end{tcolorbox}

\begin{tcolorbox}[colback=white,colframe=gray90, coltitle=black,boxrule=3pt,
fonttitle=\bfseries,title=Natural Logarithm Plus One

\begin{verbatim}
  template <short autodiff_order>
  inline var<autodiff_order> log1p(const var<autodiff_order>& input)
\end{verbatim}

\end{tcolorbox}

\begin{tcolorbox}[colback=white,colframe=gray90, coltitle=black,boxrule=3pt,
fonttitle=\bfseries,title=Natural Logarithm of One Plus Natural Exponential

\begin{verbatim}
  template <short autodiff_order>
  inline var<autodiff_order> log1p_exp(const var<autodiff_order>& input)
\end{verbatim}

\end{tcolorbox}

\begin{tcolorbox}[colback=white,colframe=gray90, coltitle=black,boxrule=3pt,
fonttitle=\bfseries,title=Binary Logarithm]

\begin{verbatim}
  template <short autodiff_order>
  inline var<autodiff_order> log2(const var<autodiff_order>& input)
\end{verbatim}

\end{tcolorbox}

\begin{tcolorbox}[colback=white,colframe=gray90, coltitle=black,boxrule=3pt,
fonttitle=\bfseries,title=Decimal Logarithm]

\begin{verbatim}
  template <short autodiff_order>
  inline var<autodiff_order> log10(const var<autodiff_order>& input)
\end{verbatim}

\end{tcolorbox}

\begin{tcolorbox}[colback=white,colframe=gray90, coltitle=black,boxrule=3pt,
fonttitle=\bfseries,title=Input Natural Logarithm Product]

\begin{verbatim}
  template <short autodiff_order>
  inline var<autodiff_order> multiply_log(const var<autodiff_order>& v1,
                                          const var<autodiff_order>& v2)
\end{verbatim}

\begin{verbatim}
  template <short autodiff_order>
  inline var<autodiff_order> multiply_log(double x,
                                          const var<autodiff_order>& v2)
\end{verbatim}

\begin{verbatim}
  template <short autodiff_order>
  inline var<autodiff_order> multiply_log(const var<autodiff_order>& v1,
                                          double y)
\end{verbatim}

\end{tcolorbox}

\subsection{Power Functions}

\begin{tcolorbox}[colback=white,colframe=gray90, coltitle=black,boxrule=3pt,
fonttitle=\bfseries,title=Cubic Root]

\begin{verbatim}
  template <short autodiff_order>
  inline var<autodiff_order> cbrt(const var<autodiff_order>& input)
\end{verbatim}

\end{tcolorbox}

\begin{tcolorbox}[colback=white,colframe=gray90, coltitle=black,boxrule=3pt,
fonttitle=\bfseries,title=Hypotenuse]

\begin{verbatim}
  template <short autodiff_order>
  inline var<autodiff_order> hypot(const var<autodiff_order>& v1,
                                   const var<autodiff_order>& v2)
\end{verbatim}

\begin{verbatim}
  template <short autodiff_order>
  inline var<autodiff_order> hypot(double x,
                                   const var<autodiff_order>& v2)
\end{verbatim}

\begin{verbatim}
  template <short autodiff_order>
  inline var<autodiff_order> hypot(const var<autodiff_order>& v1,
                                   double y)
\end{verbatim}

\end{tcolorbox}

\begin{tcolorbox}[colback=white,colframe=gray90, coltitle=black,boxrule=3pt,
fonttitle=\bfseries,title=Inverse]

\begin{verbatim}
  template <short autodiff_order>
  inline var<autodiff_order> inv(const var<autodiff_order>& input)
\end{verbatim}

\end{tcolorbox}

\begin{tcolorbox}[colback=white,colframe=gray90, coltitle=black,boxrule=3pt,
fonttitle=\bfseries,title=Power]

\begin{verbatim}
  template <short autodiff_order>
  inline var<autodiff_order> pow(const var<autodiff_order>& v1,
                                 const var<autodiff_order>& v2)
\end{verbatim}

\begin{verbatim}
  template <short autodiff_order>
  inline var<autodiff_order> pow(double x,
                                 const var<autodiff_order>& v2)
\end{verbatim}

\begin{verbatim}
  template <short autodiff_order>
  inline var<autodiff_order> pow(const var<autodiff_order>& v1,
                                 double y)
\end{verbatim}

\end{tcolorbox}

\begin{tcolorbox}[colback=white,colframe=gray90, coltitle=black,boxrule=3pt,
fonttitle=\bfseries,title=Square Root]

\begin{verbatim}
  template <short autodiff_order>
  inline var<autodiff_order> inv_sqrt(const var<autodiff_order>& input)
\end{verbatim}

\end{tcolorbox}

\begin{tcolorbox}[colback=white,colframe=gray90, coltitle=black,boxrule=3pt,
fonttitle=\bfseries,title=Square]

\begin{verbatim}
  template <short autodiff_order>
  inline var<autodiff_order> inv_square(const var<autodiff_order>& input)
\end{verbatim}

\end{tcolorbox}

\subsection{Error and Gamma Functions}

\begin{tcolorbox}[colback=white,colframe=gray90, coltitle=black,boxrule=3pt,
fonttitle=\bfseries,title=Error Function]

\begin{verbatim}
  template <short autodiff_order>
  inline var<autodiff_order> erf(const var<autodiff_order>& input)
\end{verbatim}

\end{tcolorbox}

\begin{tcolorbox}[colback=white,colframe=gray90, coltitle=black,boxrule=3pt,
fonttitle=\bfseries,title=Complimentary Error Function]

\begin{verbatim}
  template <short autodiff_order>
  inline var<autodiff_order> erfc(const var<autodiff_order>& input) 
\end{verbatim}

\end{tcolorbox}

\subsection{Other Functions}

\begin{tcolorbox}[colback=white,colframe=gray90, coltitle=black,boxrule=3pt,
fonttitle=\bfseries,title=Fast Approximation to Phi]

\begin{verbatim}
  template <short autodiff_order>
  inline var<autodiff_order> Phi_approx(const var<autodiff_order>& input)
\end{verbatim}

\end{tcolorbox}

\begin{tcolorbox}[colback=white,colframe=gray90, coltitle=black,boxrule=3pt,
fonttitle=\bfseries,title=Phi]

\begin{verbatim}
  template <short autodiff_order>
  inline var<autodiff_order> Phi(const var<autodiff_order>& input)
\end{verbatim}

\end{tcolorbox}
