\chapter{Implementation}

Code-generation and operator overloading.  References.

Existing approaches.  Reverse and forward mode types,
computation of partials on the fly.  Slooooooooow at higher-order.  

Nomad represents expression graph as a series of stacks,
storing intermediate evaluations like partial derivatives and
dual number components to optimize higher-order calculations.

\section{Architecture}

Need to represent expression graph, namely the dual numbers
at each node and the dependencies between them.

\subsection{Representing the Expression Graph With Stacks}

Once topologically sorted the nodes can be readily stored in
a linear stack, with each node storing its dual number, addresses
to inputs, and the partial derivatives.

Pictorial example of topological sort of expression graph used
in autodiff section.

Forward and reverse sweeps are then readily implemented by
running up and down the stack, propagating the dual numbers.

Only problem is that the storage will be dynamic, depending on the
function each node represents, the autodiff order, etc.  Instead
we separate out the storage into separate stacks that can more
easily handle the dynamic memory.

\subsubsection{Dual Number Stack}

Dual number multiplicities as function of autodiff order.

Storage examples for first and second order.

x1, dx1, x2, dx2

x1, dx2, dy1, d2y1, x2, dx2, dy2, d2y2.

\subsubsection{Input Stack}

Points to dual numbers of inputs nodes.

Example for two cases above.

\subsubsection{Partials Stack}

Store first, second, and third-order partial derivatives if
necessary (may not be).

Sparse storage pattern.

\subsubsection{Var Body Stack}

Pointers to dual numbers, inputs, and partials, number of inputs.

Methods for propagating perturbations and sensitivities.

Possible specializations for optimized propagation routines.

Image of architecture.

\subsection{Building the Expression Graph}

The var class.

User-facing class, points to corresponding varbody,
responsible for pushing relevant values onto each stack.

Push varbody using an arena-based memory allocation.
Allocation space for storage stacks as well, if necessary.

Leaf example.

Function example.

\section{User Interface}

Functional/functor pattern.

User defines a functor as a stateless template class implementing T operator().

Differential operations are implemented as functionals,
calling the overloaded() with the appropriate var template specialization
to build the expression graph and then sweep to compute the desired
operation.
