\chapter{Implementation}

\nomad is a \verb|C++| implementation of reverse mode automatic
differentiation that uses an \textit{operator overloading} strategy.
More precisely, we introduce a dual number type and then overload
common functions to accept this type.

A key difference of \nomad to other implementations is the generalization
to higher-orders.  Most higher-order automatic differentiation
implementations leverage the recursive nature of higher-order dual
numbers directly, automatically differentiating through a first-order
implementation to compute higher-order partial derivatives and
the subsequent linear differential operators.  This approach allows for
arbitrary-order automatic differentiation, but at the cost of inefficient and 
sometimes numerically unstable results.  \nomad,
on the other hand, explicitly implements second and third-order operators
with a focus on performance and numerical stability.

In this chapter we introduce the architecture behind the \nomad automatic 
differentiation implementation.  We view the internal representation of the 
expression graph and how it interfaces with function overloads before
discussing the user interface.

\section{Internal Representation of the Expression Graph}

While the expression graph represents the entire composite function, each
node in the expression graph represents just a single component function.  
Consequently each node must be responsible for both implementing the 
local pushforward and pullback of the corresponding function and storing 
the dual numbers, input nodes, and any auxiliary storage convenient for 
the operator implementations, like partial derivatives.
 
Like many other automatic differentiation implementations, \nomad
topologically sorts the expression graph into a linear stack of nodes, 
sometimes known as a tape (Figure \ref{fig:topologicalSort}).  This ordering 
preserves the structure of the expression graph, ensuring that a sweep 
through the stack executes a valid forward or reverse sweep through the 
expression graph.

\begin{figure}
\setlength{\unitlength}{0.1in} 
\centering
\begin{picture}(50, 20)
%
%\put(0, 0) { \framebox(50, 20){} }
%\put(25, 0) { \framebox(25, 30){} }
%\put(25, 0) { \framebox(6.25, 30){} }
%\put(25, 0) { \framebox(12.5, 30){} }
%\put(25, 0) { \framebox(18.75, 30){} }
%
%\put(25, 0) { \framebox(3.125, 30){} }
%\put(25, 0) { \framebox(9.375, 30){} }
%\put(25, 0) { \framebox(15.625, 30){} }
%
% Expression Graph
%
\put(6.25, 2.5) { \circle{4} }
\put(6.25, 2.5) { \makebox(0, 0) {$ x_{1} $} }
%
\put(12.5, 2.5) { \circle{4} }
\put(12.5, 2.5) { \makebox(0, 0) { $ x_{2} $ } }
%
\put(18.75, 2.5) { \circle{4} }
\put(18.75, 2.5) { \makebox(0, 0) { $ x_{3} $ } }
%
\put(6.25, 4.5) { \vector(3, 4){2.75} }
\put(12.5, 4.5) { \vector(-3, 4){2.75} }
\put(12.5, 4.5) { \vector(3, 4){2.75} }
\put(18.75, 4.5) { \vector(-3, 4){2.75} }
%
\put(10, 10) {\circle{4} } % Tweaked to the right
\put(9.375, 10) { \makebox(0, 0) { $f_{1}$ } }
%
\put(16.25, 10) {\circle{4} } % Tweaked to the right
\put(15.625, 10) { \makebox(0, 0) { $f_{2}$ } }
%
\put(9.375, 12) { \vector(3, 4){2.75} }
\put(15.625, 12) { \vector(-3, 4){2.75} }
%
\put(13, 17.5) {\circle{4} } % Tweaked to the right
\put(12.5, 17.5) { \makebox(0, 0) { $ g $ } }
%
% Middle Arrow
%
\put(21.875, 10) { \thicklines \vector(1, 0){6.25} }
%
% Stack
%
\put(33.5, 4) { \framebox(8, 2){ $x_{1}$} }
\put(33.5, 6) { \framebox(8, 2){ $x_{2}$ } }
\put(33.5, 8) { \framebox(8, 2){ $x_{3}$ } }
\put(33.5, 10) { \framebox(8, 2){ $f_{1}$ } }
\put(33.5, 12) { \framebox(8, 2){ $f_{2}$ } }
\put(33.5, 14) { \framebox(8, 2){ $g$ } }
%
\end{picture} 
\caption{
A topological sort of the expression graph yields a linear stack of nodes, 
sometimes known as a \textit{tape}, ordered such that a pass through the 
stack yields a valid forward or reverse sweep of the expression graph.
}
\label{fig:topologicalSort} 
\end{figure}

Nodes are implemented with the \verb|var_node| class which defines
default pushforward and pullback methods that can be specialized
for functions with structured Jacobians.  Because the amount of storage for 
each node can vary widely depending on the corresponding function and the 
order of the desired linear differential operator, storage is decoupled from 
each node.  Instead the necessary data is stored in three global stacks, the 
inputs stack, dual numbers stack, and partials stack, with \verb|var_node| 
containing only an address to the relevant data in each 
(Figure \ref{fig:architecture}) as well as accessor and mutator methods that 
abstract the indirect storage.

\begin{figure}
\setlength{\unitlength}{0.1in} 
\centering
\begin{picture}(50, 30)
%
%\put(0, 0) { \framebox(50, 30){} }
%
%\put(0, 0) { \framebox(12.5, 30){} }
%\put(0, 0) { \framebox(25, 30){} }
%\put(0, 0) { \framebox(37.5, 30){} }
%
%\put(0, 0) { \framebox(50, 7.5){} }
%\put(0, 0) { \framebox(50, 15){} }
%\put(0, 0) { \framebox(50, 22.5){} }
%
% Var Body
%
\put(8.5, 2) { \makebox(8, 2){Var Node} }
\put(8.5, 6) { \framebox(8, 2){} }
\put(8.5, 8) { \framebox(8, 2){} }
\put(8.5, 10) { \framebox(8, 2){} }
\put(8.5, 12) { \framebox(8, 2){} }
\put(8.5, 14) { \framebox(8, 2){} }
\put(8.5, 16) { \framebox(8, 2){} }
\put(8.5, 18) { \framebox(8, 2){} }
\put(8.5, 20) { \framebox(8, 2){} }
\put(8.5, 22) { \framebox(8, 2){} }
\put(8.5, 24) { \framebox(8, 2){} }
%
% Inputs
%
\put(39, 24) { \makebox(8, 2){Inputs} }
\put(25, 22) { \framebox(2, 6){} }
\put(27, 22) { \framebox(2, 6){} }
\put(29, 22) { \framebox(2, 6){} }
\put(31, 22) { \framebox(2, 6){} }
\put(33, 22) { \framebox(2, 6){} }
\put(35, 22) { \framebox(2, 6){} }
\put(37, 22) { \framebox(2, 6){} }
%
% Dual Numbers
%
\put(39, 15) { \makebox(8, 2){Dual} }
\put(39, 13) { \makebox(8, 2){Numbers} }
\put(25, 12) { \framebox(2, 6){} }
\put(27, 12) { \framebox(2, 6){} }
\put(29, 12) { \framebox(2, 6){} }
\put(31, 12) { \framebox(2, 6){} }
\put(33, 12) { \framebox(2, 6){} }
\put(35, 12) { \framebox(2, 6){} }
\put(37, 12) { \framebox(2, 6){} }
%
% Partials
%
\put(39, 4) { \makebox(8, 2){Partials} }
\put(25, 2) { \framebox(2, 6){} }
\put(27, 2) { \framebox(2, 6){} }
\put(29, 2) { \framebox(2, 6){} }
\put(31, 2) { \framebox(2, 6){} }
\put(33, 2) { \framebox(2, 6){} }
\put(35, 2) { \framebox(2, 6){} }
\put(37, 2) { \framebox(2, 6){} }
%
% Arrows
%
\thicklines
\put(32, 22) { \vector(-1, -1){4} }
\put(34, 22) { \vector(0, -1){4} }
%
\put(16.5, 15) { \vector(4, 3){15.5} }
\put(16.5, 15) { \vector(1, 0){19.5} }
\put(16.5, 15) { \vector(2, -1){17.5} }
%
\end{picture} 
\caption{ 
Upon a topological sort (Figure \ref{fig:topologicalSort}), the expression graph 
is represented by a stack of \texttt{var\_node} objects.  The input nodes, 
dual numbers, and partial derivatives are stored in external stacks, with each 
\texttt{var\_node} storing only addresses to each.  Note that the inputs stack 
address not the \texttt{var\_node} objects directly, but rather only the dual numbers 
of those nodes needed for implementing the pushforward and pullback operators.
}
\label{fig:architecture} 
\end{figure}

\subsubsection{The Inputs Stack}

The inputs stack represents the edges in the expression graph
needed for directing the forward and reverse sweeps.  Because
the pushforward and pullback operators need to read and write only 
the dual numbers at each node, edges index the dual numbers of any
dependencies directly (Figure \ref{fig:architecture}) and avoid the overhead
that would be acquired for indirect access through the \verb|var_node|
objects.

\subsubsection{Dual Number Stack}

The dual number stack stores the $2^{k}$ components of the $k$-order
dual numbers at each node.  For example, a first-order expression graph
will need only two elements for each node while a third-order expression 
graph will need eight (Figure \ref{fig:dualNumberStorage}).

\begin{figure}
\setlength{\unitlength}{0.1in} 
\centering
\begin{picture}(50, 20)
%
%\put(0, 0) { \framebox(50, 20){} }
%
% First-Order
%
\put(0, 15) { \makebox(8, 2){First-} }
\put(0, 13) { \makebox(8, 2){Order} }
\put(9, 17) { \makebox(2, 2){ $\ldots$ } }
\put(9, 11) { \makebox(2, 2){ $\ldots$ } }
\put(11, 12) { \framebox(4, 6){ $x $ } }
\put(15, 12) { \framebox(4, 6){ $ \delta x $ } }
\put(19, 17) { \makebox(2, 2){ $\ldots$ } }
\put(19, 11) { \makebox(2, 2){ $\ldots$ } }
%
{ \thicklines \put(13, 20) { \vector(0, -1){2} } }
%
% Second-Order
%
\put(0, 5) { \makebox(8, 2){Third-} }
\put(0, 3) { \makebox(8, 2){Order} }
\put(9, 7) { \makebox(2, 2){ $\ldots$ } }
\put(9, 1) { \makebox(2, 2){ $\ldots$ } }
\put(11, 2) { \framebox(4, 6){ $s$ } }
\put(15, 2) { \framebox(4, 6){ $\delta s$ } }
\put(19, 2) { \framebox(4, 6){ $\delta t$ } }
\put(23, 2) { \framebox(4, 6){ $\delta^{2} t$ } }
\put(27, 2) { \framebox(4, 6){ $\delta u$ } }
\put(31, 2) { \framebox(4, 6){ $\delta^{2} u$ } }
\put(35, 2) { \framebox(4, 6){ $\delta^{2} v$ } }
\put(39, 2) { \framebox(4, 6){ $\delta^{3} v$ } }
\put(43, 7) { \makebox(2, 2){ $\ldots$ } }
\put(43, 1) { \makebox(2, 2){ $\ldots$ } }
%
{ \thicklines \put(13, 10) { \vector(0, -1){2} } }
%
\end{picture} 
\caption{
The dual number stack is able to accommodate storage of dual numbers 
of any order without reallocating memory.  For example, first-order dual 
numbers require only two elements while third-order dual numbers require
eight elements.
}
\label{fig:dualNumberStorage} 
\end{figure}

\subsubsection{Partials Stack}

A common approach in many automatic differentiation implementations
is to compute partial derivatives as necessary and avoid storing them
in the expression graph.  This strategy is sound for first-order reverse
mode calculations where the partials are used only once, but higher-order
calculations require multiple sweeps that reuse the partials.  Recomputing
the partials for each sweep becomes a significant computational burden.

With a focus on implementing efficient higher-order methods, \nomad 
explicitly stores partial derivatives in a dedicated partials stack.  When 
constructing an $k$-order expression graph only the $k$-order partial 
derivatives and lower are stored, and only if the partial derivatives are 
non-zero.  For example, a second-order expression graph will calculate 
and store only the first and second-order partial derivatives.

To avoid redundant calculations and storage, \nomad leverages the 
stores only the unique higher-order values.  In a second-order calculation 
a node representing the component function 
$f: \mathbb{R}^{N} \rightarrow \mathbb{R}$ will store
%
\begin{equation*}
\frac{ \partial^{2} f }{ \partial x_{i} \partial x_{j} }, \, i \in 1, \ldots, N, j \in 1, \ldots i,
\end{equation*}
%
while a third-order calculation requires
%
\begin{equation*}
\frac{ \partial^{3} f }{ \partial x_{i} \partial x_{j} \partial x_{k} }, \, 
i \in 1, \ldots, N, j \in 1, \ldots i, k \in 1, \ldots, j.
\end{equation*}
%
For example, when executing a third-order calculation the node representing
a binary function, $f : \mathbb{R}^{2} \rightarrow \mathbb{R}$, utilizes the
compact storage demonstrated in Figure \ref{fig:partialsStorage}.

\begin{figure}
\setlength{\unitlength}{0.1in} 
\centering
\begin{picture}(50, 10)
%
%\put(0, 0) { \framebox(50, 10){} }
%
% Second-Order
%
\put(5, 7) { \makebox(2, 2){ $\ldots$ } }
\put(5, 1) { \makebox(2, 2){ $\ldots$ } }
\put(7, 2) { \framebox(4, 6){ $ \frac{ \partial f }{ \partial x} $ } }
\put(11, 2) { \framebox(4, 6){ $ \frac{ \partial f }{ \partial y} $ } }
\put(15, 2) { \framebox(4, 6){ $ \frac{ \partial^{2} f }{ \partial x^{2}} $ } }
\put(19, 2) { \framebox(4, 6){ $ \frac{ \partial^{2} f }{ \partial x \partial y} $ } }
\put(23, 2) { \framebox(4, 6){ $ \frac{ \partial^{2} f }{ \partial y^{2}} $ } }
\put(27, 2) { \framebox(4, 6){ $ \frac{ \partial^{3} f }{ \partial x^{3}} $ } }
\put(31, 2) { \framebox(4, 6){ $ \frac{ \partial^{3} f }{ \partial x^{2} \partial y} $ } }
\put(35, 2) { \framebox(4, 6){ $ \frac{ \partial^{3} f }{ \partial x \partial y^{3}} $ } }
\put(39, 2) { \framebox(4, 6){ $ \frac{ \partial^{3} f }{ \partial y^{3}} $ } }
\put(43, 7) { \makebox(2, 2){ $\ldots$ } }
\put(43, 1) { \makebox(2, 2){ $\ldots$ } }
%
{ \thicklines \put(9, 10) { \vector(0, -1){2} } }
%
\end{picture} 
\caption{
Only unique partial derivatives are stored in the partials stack,
as demonstrated here for node representing a binary function,
$f: \mathbb{R}^{2} \rightarrow \mathbb{R}$, in a third-order 
expression graph that requires first, second, and third-order
partial derivatives.
}
\label{fig:partialsStorage} 
\end{figure}

In general a function with $N$ inputs and non-vanishing derivatives at
all orders will require $\binom{N + M}{M}$ elements to store the unique
$M$-order partial derivatives.

\section{Extending Functions to Accept Dual Numbers}

In order to implement reverse mode calculations, each dual-number 
valued-function is responsible for not only computing the function
value but also expanding the expression graph with the addition of
a node with the desired pushforward and pullback methods and
addresses to the top of the inputs, dual numbers, and partials stacks.
The evaluation of a composite dual number-valued function will then
build the complete expression graph primed for the application of
linear differential operators.

Dual numbers in \nomad are exposed to the user by the \verb|var|
class with \verb|var|-valued functions responsible for the management
of the expression graph.  In this section we present the details of
the \verb|var| class and the implementation of \verb|var|-valued functions.

\subsection{The var Class}

The \verb|var| class itself is only a lightweight wrapper for an underlying 
\verb|var_node| in the expression graph, storing an address to the 
corresponding node with a variety of member functions that expose the 
node's data.  Most importantly, \verb|var| is a templated is allow for
the compile-time configuration of the expression graph,
%
\begin{verbatim}
template <short AutodiffOrder, bool StrictSmoothness, bool ValidateIO>
class var { ...
\end{verbatim}
%
\textbf{\nomad also defines the typdefs...}

\begin{table*}[t!]
	\centering
	\renewcommand{\arraystretch}{2}
	\begin{tabular}{cccc}
	\rowcolor[gray]{0.9} Type Definition & \verb|AutodiffOrder|
	& \verb|StrictSmoothness| & \verb|ValidateIO| \\
	\verb|var1| & \verb|1| & \verb|true| & \verb|false| \\
	\rowcolor[gray]{0.9} \verb|var2| & \verb|2| & \verb|true| & \verb|false| \\
	\verb|var3| & \verb|3| & \verb|true| & \verb|false| \\
	\rowcolor[gray]{0.9} \verb|var1_debug| & \verb|1| & \verb|true| & \verb|true| \\
	\verb|var2_debug| & \verb|2| & \verb|true| & \verb|true| \\
	\rowcolor[gray]{0.9} \verb|var3_debug| & \verb|3| & \verb|true| & \verb|true| \\
	\verb|var1_wild| & \verb|1| & \verb|false| & \verb|false| \\
	\rowcolor[gray]{0.9} \verb|var2_wild| & \verb|2| & \verb|false| & \verb|false| \\
	\verb|var3_wild| & \verb|3| & \verb|false| & \verb|false| \\
	\end{tabular}
	\label{tab:typedefs}
	\caption{Some typedefs!.}
\end{table*}

\subsubsection{AutodiffOrder}

The template parameter \verb|AutodiffOrder| defines the maximum order
linear differential operator that the corresponding expression graph will
admit.  Attempting to apply a second-order linear differential operator
to a graph built from \verb|var|s with \verb|AutodiffOrder = 1|, for example,
will induce a compile-time error.

Note that lower-order graphs require the computation and storage of
fewer dual numbers and partial derivatives, not to mention faster pushforward
and pullback implementations.  Consequently, although a higher-order
graph will admit lower-order operators it is much more efficient to match
the order of the expression graph, via the choice of \verb|AutodiffOrder|,
to the linear differential operator of interest.

\subsubsection{StrictSmoothness}

\verb|StrictSmoothness| defines the smoothness requirements of 
component functions.

Formally, automatic differentiation requires only that each component
function have well-behaved derivatives defined in some neighborhood
containing the point at which the function is being evaluated.  For
example, a linear differential operator applied to the absolute value
function is well defined everywhere except at the origin.  Many
algorithms that use these operators, however, actually require that
the derivatives be well-defined \textit{everywhere} because even 
functions with only point discontinuities in their derivatives will
manifest undesired pathologies.

With these algorithms in mind, when \verb|StrictSmoothness = true| 
\nomad will disable functions and operators that are not everywhere 
smooth, as well as those that admit comparisons that may introduce cusps.

\subsubsection{ValidateIO}

\verb|ValidateIO| enables strict validation of the inputs and outputs
to each component function.

Identifying the source of potential floating-point pathologies like
underflow, overflow and \verb|NaN| in a large, composite function
is often a challenging debugging task.  When \verb|ValidateIO = true|,
\nomad assists users by checking all input values, output values,
and output partial derivatives for \verb|NaN| and throws an
exception identifying the responsible component function if any
are found.  Domain validation will also be done if implemented in
the function.  These checks, however, are a nontrivial computational
burden and should be disabled for high-performance applications.

\subsection{var-valued Functions}

Validate inputs using validate methods.

\begin{verbatim}
inline void validate_input(double val, std::string f_name)
\end{verbatim}

\begin{verbatim}
inline void validate_lower_bound(double val, double lower, std::string f_name)
\end{verbatim}

Push inputs onto stack using push\_input method

\begin{verbatim}
inline void push_inputs(nomad_idx_t input)
\end{verbatim}

Compute function value and push dual numbers onto stack
using push\_dual\_numbers method.  May throw an exception
if ValidateIO is true and so should be wrapped in a \verb|try/catch| block.

\begin{verbatim}
template<short AutodiffOrder, bool ValidateIO>
inline void push_dual_numbers(double val)
\end{verbatim}

Compute partials onto stack using push\_partials method.
May throw an exception if ValidateIO is true and so should be wrapped 
in a \verb|try/catch| block.

\begin{verbatim}
template<bool ValidateIO>
inline void push_partials(double partial)
\end{verbatim}


\subsubsection{Smooth Functions}

\begin{verbatim}
template <short AutodiffOrder, bool StrictSmoothness, bool ValidateIO>
inline var<AutodiffOrder, StrictSmoothness, ValidateIO>
  binary_function(const var<AutodiffOrder, StrictSmoothness, ValidateIO>& v1,
                  const var<AutodiffOrder, StrictSmoothness, ValidateIO>& v2) {
    
  // Validate input values if ValidateIO is true
  if (ValidateIO) {
    validate_input(v1.first_val(), "binary_function");
    validate_input(v2.first_val(), "binary_function");
  }
      
  // Create a var_node on the top of the stack
  const short partials_order = 3;
  const unsigned int n_inputs = 2;
    
  create_node<var_node<AutodiffOrder, partials_order>>(n_inputs);

  // Push dependencies to top of the inputs stack
  push_inputs(v1.dual_numbers());
  push_inputs(v2.dual_numbers());

  // Push dual numbers to the the top of the dual numbers stack
  double x = v1.first_val();
  double y = v2.first_val();
    
  try {
    push_dual_numbers<AutodiffOrder, ValidateIO>(binary_function(x, y));
  } catch(nomad_error& e) {
    throw nomad_output_value_error("binary_function");
  }
    
  // Push partial derivatives to the top of the partials stack
  try {
    if (AutodiffOrder >= 1) {
      push_partials<ValidateIO>(df_dx);
      push_partials<ValidateIO>(df_dy);
    }
    if (AutodiffOrder >= 2) {
      push_partials<ValidateIO>(df2_dx2);
      push_partials<ValidateIO>(df2_dxdy);
      push_partials<ValidateIO>(df2_dy2);
    }
    if (AutodiffOrder >= 3) {
      push_partials<ValidateIO>(df3_dx3);
      push_partials<ValidateIO>(df3_dx2dy);
      push_partials<ValidateIO>(df3_dxdy2);
      push_partials<ValidateIO>(df3_dy3);
    }
  } catch(nomad_error& e) {
    throw nomad_output_partial_error("binary_function");
  }

  // Return a new var that wraps the newly created node
  return var<AutodiffOrder, StrictSmoothness, ValidateIO>(next_node_idx_ - 1);
    
}
\end{verbatim}

Note use of template parameters to avoid expensive computations
whenever possible.  Other relevant discussion.

Example of a \verb|var_node| specialization.

\begin{verbatim}
template <short AutodiffOrder, bool StrictSmoothness, bool ValidateIO>
inline var<AutodiffOrder, StrictSmoothness, ValidateIO>
  operator+(const var<AutodiffOrder, StrictSmoothness, ValidateIO>& v1,
            const var<AutodiffOrder, StrictSmoothness, ValidateIO>& v2) {

  if (ValidateIO) {
    validate_input(v1.first_val(), "operator+");
    validate_input(v2.first_val(), "operator+");
  }
      
  // Create a specialized var_node on the stack
  create_node<binary_sum_var_node<AutodiffOrder>>(2);
    
  push_inputs(v1.dual_numbers());
  push_inputs(v2.dual_numbers());
    
  try {
    push_dual_numbers<AutodiffOrder, ValidateIO>(v1.first_val() + v2.first_val());
  } catch(nomad_error& e) {
    throw nomad_output_value_error("operator+");
  }
    
  // The binary_sum_var_node pushforward and pullback implementations
  // use hardcoded partial derivatives and don't require partial derivatives to 
  // be pushed onto the partials stack
    
  return var<AutodiffOrder, StrictSmoothness, ValidateIO>(next_node_idx_ - 1);
    
}
\end{verbatim}

\subsubsection{Non-Smooth Functions}

Just show enable-if'd signature.

\begin{verbatim}
template <short AutodiffOrder, bool StrictSmoothness, bool ValidateIO>
inline typename 
  std::enable_if<!StrictSmoothness, 
                 var<AutodiffOrder, StrictSmoothness, ValidateIO> >::type
  fabs(const var<AutodiffOrder, StrictSmoothness, ValidateIO>& input) {
    
  if (ValidateIO) validate_input(input.first_val(), "fabs");
      
  const short partials_order = 1;
  const unsigned int n_inputs = 1;
    
  create_node<unary_var_node<AutodiffOrder, partials_order>>(n_inputs);

  double x = input.first_val();
    
  try {
    push_dual_numbers<AutodiffOrder, ValidateIO>(fabs(x));
  } catch(nomad_error& e) {
    throw nomad_output_value_error("fabs");
  }
      
  push_inputs(input.dual_numbers());
    
  try {
    if (AutodiffOrder >= 1) {
      if (x < 0)
        push_partials<ValidateIO>(-1);
      else
        push_partials<ValidateIO>(1);
    }
  } catch(nomad_error& e) {
    throw nomad_output_partial_error("fabs");
  }

  return var<AutodiffOrder, StrictSmoothness, ValidateIO>(next_node_idx_ - 1);
    
}
\end{verbatim}

\section{The \nomad User Interface}

Functor-functional relationship.

Implement base functor with operator().  Example.

\begin{verbatim}
template<class T>
class base_functor {
public:
  virtual T operator()(const Eigen::VectorXd& x) const;
  typedef T var_type;
};
\end{verbatim}

\begin{equation*}
f \! \left( x, y, z \right) = \cos \! \left( e^{x} + e^{y} \right) / z
\end{equation*}
%
would be defined as
%
\begin{verbatim}
template <typename T>
class example_functor: public base_functor<T> {
  T operator()(const Eigen::VectorXd& x) const {
    T v1 = x[0];
    T v2 = x[1];
    T v3 = x[2];
    return cos( exp(v1) + exp(v2) ) / v3;  
  }
};
\end{verbatim}

Compute linear differential operator by
passing instance of functor to function.  Functionals
evaluate functor to build expression graph, then
propagate execute forward and reverse sweeps
as necessary.  Gradient example.
%
\begin{verbatim}
  template <typename F>
  void gradient(const F& f,
                const Eigen::VectorXd& x,
                Eigen::VectorXd& g)
\end{verbatim}

\begin{verbatim}
int n = 100;
Eigen::VectorXd x = Eigen::VectorXd::Ones(n);
  
double f;
Eigen::VectorXd grad(x.size());
  
funnel_func<var1> example;
gradient(example, x, f, grad);
\end{verbatim}
